%Copyright 2019 Rice University
%
%Licensed under the Apache License, Version 2.0 (the "License");
%you may not use this file except in compliance with the License.
%You may obtain a copy of the License at
%
%    https://www.apache.org/licenses/LICENSE-2.0
%
%Unless required by applicable law or agreed to in writing, software
%distributed under the License is distributed on an "AS IS" BASIS,
%WITHOUT WARRANTIES OR CONDITIONS OF ANY KIND, either express or implied.
%See the License for the specific language governing permissions and
%limitations under the License.
\documentclass[11pt]{article}
%\documentclass[12pt]{amsart}
%\usepackage{latex8}
\usepackage{fullpage}
\usepackage{times}
\usepackage{url}
\usepackage[normalem]{ulem}
\usepackage{epsfig} 
%\usepackage{latexsym}
\usepackage{subfigure}
\usepackage{graphicx}
\usepackage{titlesec}
\usepackage{multirow}

\titlespacing{\section}{0pt}{3mm}{1mm}
\titlespacing{\subsection}{0pt}{2mm}{0.5mm}
\titlespacing{\subsubsection}{0pt}{2mm}{0.8mm}

%\topmargin 0.75in 
%\oddsidemargin -0.04in
%\textwidth 6.5in
%\textheight 9.0in 
%\setlength{\textheight}{23.1cm}
%\setlength{\textwidth}{17.0cm}

\newcommand{\muhat}{\hat{\mu}}
\newcommand{\sigmahat}{\hat{\sigma}}
\newcommand{\todo}[1]{[\textbf{TODO: #1}]}
\newcommand{\eat}[1]{} % TO MAKE LARGE BLOCKS OF TEXT INVISIBLE
\newcommand{\sz}[1]{\lvert#1\rvert}
\newcommand{\card}[1]{\lvert#1\rvert}
\newcommand{\xp}[2]{P \if*#1\else^{#1}\fi \if*#2\else_{\! #2}\fi}
\newcommand{\pr}[3]{\xp{#1}{#2}\left\{\,#3\,\right\}}
\newcommand{\prl}[3]{\xp{#1}{#2}\{\,#3\,\}}
\renewcommand\:{\colon} % for use with \sset, etc.
\newcommand{\sset}[1]{\left\{\,#1\,\right\}}
\newcommand\xD{\mathcal{D}}
\newcommand\xP{\mathcal{P}}
\newcommand\xS{\mathcal{S}}
\newcommand\xbar{\bar x}
\newcommand\vbar{\bar v}
\newcommand\xmax{{x_{\text{max}}}}
\newcommand\eps{\epsilon}
\newcommand{\eeblk}{\hbox{\lower 1pt \vbox{\hrule width6pt\hbox to
  6pt{\vrule height5pt depth1pt \hfil\vrule height5pt depth1pt} \hrule
  width6pt} \unskip}}
\newcommand{\eblk}{{\unskip\nobreak\hfil\penalty50
  \hskip1em\hbox{}\nobreak\hfil\eeblk
  \parfillskip=0pt\finalhyphendemerits=0\par}}
\newtheorem{xample}{Example}
%\newenvironment{example}{\begin{xample}\em}{\eblk\end{xample}}
\makeatletter
\newenvironment{sql}%
 {\vskip 5pt\begin{list}{}{%
  \setlength{\topsep}{0pt}\setlength{\partopsep}{0pt}\setlength{\parskip}{0pt}%
  \setlength{\parsep}{0pt}\setlength{\labelwidth}{0pt}%
  \setlength{\rightmargin}{0pt}\setlength{\leftmargin}{0pt}%
  \setlength{\labelsep}{0pt}%
  \obeylines\@vobeyspaces\normalfont\ttfamily%
  \item[]}}
 {\end{list}\vskip5pt\noindent}
\makeatother
\newcommand{\bpar}[1]{\vskip 5pt\noindent\textbf{#1}\hskip 1em}
\newcommand\yN{{\tilde N}}
\newcommand\yX{{\tilde X}}
\newcommand\ymu{{\tilde\mu}}
\newcommand\ysigma{{\tilde\sigma}}


\newcommand{\goodgap}{
        \hspace{\subfigtopskip}
        \hspace{\subfigbottomskip}
}

%\renewcommand{\baselinestretch}{0.99}

\newtheorem{definition}{Definition}
\newtheorem{Rule}{Rule}
\newtheorem{lemma}{Lemma}
\newtheorem{theorem}{Theorem}
\newtheorem{problem}{Problem}
\newtheorem{example}{Example}
\newtheorem{optimization}{Optimization}
\newtheorem{observation}{Observation}
\newtheorem{corollary}{Corollary}

\newcommand{\qed}{\hspace*{\fill}
           \vbox{\hrule\hbox{\vrule\squarebox{.667em}\vrule}\hrule}\smallskip}

\long\def \ignoreme#1{}

\def\qed{\hfill \mbox{\rule[0pt]{1.5ex}{1.5ex}}}



\begin{document}
%\maketitle
%\pagestyle{empty}

\begin{center}
{\bf \huge{COMP 543 Lab \#1: Setting up Jupyter Notebooks}}
\end{center}


\section{Download}
\noindent Download the Python 3.6 version from:
\url{https://www.anaconda.com/download} and install.
\vspace{1em}

\noindent You may also install Python 3 \url{www.python.org} and Jupyter \url{jupyter.org} without installing Anaconda.
\vspace{1em}


\noindent Install using the defaults.

\vspace{1em}

\section{Install packages}
\noindent To integrate PostgreSQL and Jupyter (we will install Postgres in a later lab):
\begin{enumerate}
\item Run Anaconda with administrative privileges
\item Select ``All" from the drop down at the top left of the main pane.
\item Search for psycopg2 and check the box next to it. 
\noindent For more information on this package, see \url{ https://pypi.python.org/pypi/psycopg2 }
\item Click Apply
\item Allow any dependencies to be installed as well.
\end{enumerate}

Also install the ipython-sql package using one of the following commands:
\begin{verbatim}
pip install ipython-sql
python3.6 -m pip install ipython-sql
\end{verbatim}

\noindent For more information on ipython-sql, see \url{pypi.python.org/pypi/ipython-sql }.
%--------------------------------------------------
\section{Set up Jupyter}
\begin{enumerate}
\item Choose a directory that will contain the course notebooks, such as:
\begin{verbatim}
/Users/rbm2/Documents/School/Rice/DSToolsAndModels/notebooks
\end{verbatim}
\item Create a configuration file
\begin{verbatim}
jupyter notebook --generate-config
\end{verbatim}
\item Find your configuration file (the previous step should tell you where it is). Typically it is your home directory, in a new folder named .jupyter. The configuration file will be named \texttt{jupyter\_notebook\_config.py}.
\item Edit the configuration file to point to the default notebook directory.
\begin{enumerate}
\item Find the row that says: \texttt{\#c.NotebookApp.notebook\_dir = ''}
\item Uncomment it and include the path to your notebook folder\\
\small{\texttt{c.NotebookApp.notebook\_dir = '/Users/rbm2/Documents/School/Rice/DSToolsAndModels/notebooks'}}
\end{enumerate}
\end{enumerate}


\end{document}
