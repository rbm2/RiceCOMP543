%Copyright 2019 Christopher M. Jermaine (cmj4@rice.edu) and Risa B. Myers (rbm2@rice.edu)
%
%Licensed under the Apache License, Version 2.0 (the "License");
%you may not use this file except in compliance with the License.
%You may obtain a copy of the License at
%
%    https://www.apache.org/licenses/LICENSE-2.0
%
%Unless required by applicable law or agreed to in writing, software
%distributed under the License is distributed on an "AS IS" BASIS,
%WITHOUT WARRANTIES OR CONDITIONS OF ANY KIND, either express or implied.
%See the License for the specific language governing permissions and
%limitations under the License.
%
%===============================================================%
\documentclass[aspectratio=169]{beamer}

\mode<presentation> 
{
\usetheme[noshadow, minimal,numbers,riceb,nonav]{Rice}
\usefonttheme[onlymath]{serif}

\setbeamercovered{transparent}
}
\useinnertheme{rectangles}
\usepackage{colortbl}

\usepackage[english]{babel}

\usepackage{mathptmx}
\usepackage{helvet}
\usepackage{courier}
\usepackage[T1]{fontenc}
\usepackage{trajan}
\usepackage{ textcomp }
\usepackage{amssymb}

%https://tex.stackexchange.com/questions/20740/symbols-for-outer-joins
\def\ojoin{\setbox0=\hbox{$\bowtie$}%
  \rule[-.02ex]{.25em}{.4pt}\llap{\rule[\ht0]{.25em}{.4pt}}}
\def\leftouterjoin{\mathbin{\ojoin\mkern-5.8mu\bowtie}}
\def\rightouterjoin{\mathbin{\bowtie\mkern-5.8mu\ojoin}}
\def\fullouterjoin{\mathbin{\ojoin\mkern-5.8mu\bowtie\mkern-5.8mu\ojoin}}


\usepackage{listings}

\newenvironment{noindentitemize}
{ \begin{itemize}
 \setlength{\itemsep}{1.5ex}
  \setlength{\parsep}{0pt}   
  \setlength{\parskip}{0pt}
 \addtolength{\leftskip}{-2em}
 }
{ \end{itemize} }

\newenvironment{noindentitemize2}
{ \begin{itemize}
  \setlength{\itemsep}{0ex}
  \setlength{\parskip}{0pt}
  \setlength{\parsep}{0pt}   
  \addtolength{\leftskip}{-2em}  }
{ \end{itemize} }

\lstnewenvironment{SQL}
  {\lstset{
        aboveskip=5pt,
        belowskip=5pt,
        escapechar=!,
        mathescape=true,
        upquote=true,
        language=SQL,
        basicstyle=\linespread{0.94}\ttfamily\footnotesize,
        morekeywords={FOR, EACH, WITH, PARTITION, AND, ALL, TEST, WHETHER, PROBABILITY},
        deletekeywords={VALUE, PRIOR},
        showstringspaces=true}
        \vspace{0pt}%
        \noindent\minipage{0.75\textwidth}}
  {\endminipage\vspace{0pt}}


\newcommand{\ALL}{\texttt{ALL}} 
\newcommand{\LIKES}{\textrm{LIKES}} 
\newcommand{\FREQUENTS}{\textrm{FREQUENTS}} 
\newcommand{\SERVES}{\textrm{SERVES}} 
\newcommand{\CAFE}{\textrm{CAFE}} 
\newcommand{\COFFEE}{\textrm{COFFEE}} 
\newcommand{\DRINKER}{\textrm{DRINKER}} 
\newcommand{\CB}{\textrm{\textquotesingle{Cold} Brew\textquotesingle}} 
\newcommand{\CBGOOD}{\textrm{CBGOOD}} 
\newcommand{\ALLPEEPS}{\textrm{ALLPEEPS}} 
\newcommand{\ALLCOMBOS}{\textrm{ALLCOMBOS}} 
\newcommand{\NOGOODCOFFEE}{\textrm{NOGOODCOFFEE}} 

\setbeamerfont{block body}{size=\tiny}

%===============================================================%

\title[]
{Tools \& Models for Data Science}


\subtitle{SQL 1}

\author[]{Chris Jermaine \& Risa Myers}
\institute
{
  Rice University
}

\date[]{}

\begin{document}
		
\begin{frame}
 \titlepage
\end{frame}

%***********************************************************

\begin{frame}{SQL}

\begin{noindentitemize}
\item De-facto standard DB programming language
	\begin{noindentitemize2}
	\item First proposed by IBM researchers in 1970's
	\item Oracle first to offer commercial version in 1979
	\item IBM soon after
	\end{noindentitemize2}
\end{noindentitemize}
\end{frame}

%***********************************************************
\begin{frame}{SQL}

\begin{noindentitemize}
\item SQL is a H U G E language!!
	\begin{noindentitemize2}
	\item Current standard runs to 100s of pages
	\item Consists of a declarative DML
	\item And an imperative DML
	\item And a DDL
	\end{noindentitemize2}
\item We begin with the heart and soul of SQL: the declarative DML
\end{noindentitemize}
\end{frame}
%***********************************************************
\begin{frame}{Relational Algebra vs. SQL}

\begin{itemize}
\item Duplicates are not automatically eliminated (Multi-sets vs. sets)
\item Not all SQL implementations support all RA operators
\begin{itemize}
\item e.g. Difference operator
\end{itemize}
\item SQL extends RA
\begin{itemize}
\item with aggregate functions
\item with schema modifications
\end{itemize}
\end{itemize}

\end{frame}
%***********************************************************
\begin{frame}{Relational Algebra Operators to SQL}

\begin{tabular}{|l|l|l|} \hline
RA Name & RA symbol & SQL term \\ \hline
Projection & $\pi$ & SELECT [L: attribute list] \\\hline
Join & $\times \ \bowtie *$ & FROM [R: Relation list] \\ \hline
Selection & $\sigma$ & WHERE [C: Condition list] \\ \hline
\end{tabular}


\vspace{2em}
{\Large 
 $\pi_L(\sigma_C(R))$}
\end{frame}

%***********************************************************
\begin{frame}{Relational Algebra Operators to SQL}

\begin{tabular}{|l|l|l|} \hline
RA Name & RA Symbol &  SQL equivalent \\ \hline
Union & $\cup$ & UNION or UNION ALL \\\hline
Intersection & $\cap$ & JOIN or EXISTS or IN \\ \hline
Difference & $-$ & JOIN with NULL test or EXCEPT \\ \hline
Rename & $\rho$ & AS \\ \hline
Assignment & $\leftarrow$ & INTO \\ \hline
\end{tabular}

\end{frame}


%***********************************************************
\begin{frame}[fragile]{Our First Query}
\begin{itemize}
\item Given the following relations:
\end{itemize}


LIKES (DRINKER, COFFEE)

FREQUENTS (DRINKER, CAFE)

SERVES (CAFE, COFFEE)

\begin{itemize}
\item Who goes to a cafe serving Cold Brew?  %(\textquotesingle{}EK\textquotesingle{})
\end{itemize}

\begin{SQL}
SELECT

FROM

WHERE
\end{SQL}
\end{frame}

%***********************************************************
\begin{frame}[fragile]{Our First Query}
\begin{noindentitemize}
\item Given the following relations:
\end{noindentitemize}
	

LIKES (DRINKER, COFFEE)

FREQUENTS (DRINKER, CAFE)
SERVES (CAFE, COFFEE)

\begin{noindentitemize}
\item Who goes to a cafe serving Cold Brew?  %(\textquotesingle{}EK\textquotesingle{})
\end{noindentitemize}

\begin{SQL}
SELECT DISTINCT f.DRINKER
FROM FREQUENTS AS f, SERVES AS s
WHERE f.CAFE = s.CAFE AND s.COFFEE = 'Cold Brew'
\end{SQL}

\begin{noindentitemize}
\item[?] What happens without \texttt{DISTINCT}? % we get repeats of drinkers who frequent more than 1 cafe
\end{noindentitemize}
\end{frame}
	
%***********************************************************
\begin{frame}[fragile]{Our First Query}

LIKES (DRINKER, COFFEE)

FREQUENTS (DRINKER, CAFE)

SERVES (CAFE, COFFEE)

\begin{itemize}
\item Who goes to a cafe serving Cold Brew?  %(\textquotesingle{}EK\textquotesingle{})
\end{itemize}

\begin{SQL}
SELECT DISTINCT f.DRINKER
FROM FREQUENTS AS f, SERVES AS s
WHERE f.CAFE = s.CAFE AND s.COFFEE = 'Cold Brew'
\end{SQL}

\begin{itemize}
\item Closely related to RC!  Same as:
\item[] $\{f.\DRINKER | \FREQUENTS(f) \wedge \SERVES(s) $\\
\hspace{2em}$\wedge\ f.\CAFE = s.\CAFE \wedge s.\COFFEE = 
		\CB\}$
\end{itemize}
\end{frame}

%***********************************************************
\begin{frame}[fragile]{AS}

\begin{SQL}
SELECT DISTINCT f.DRINKER
FROM FREQUENTS AS f, SERVES AS s
WHERE f.CAFE = s.CAFE AND s.COFFEE = 'Cold Brew'
\end{SQL}

\begin{itemize}
\item What does \texttt{AS} do?
\begin{itemize}
\item Rename ($\rho$) from Relational Algebra!
\item Works on tables as well as attributes
\item Actual key word is optional
\item Why bother? To create a more meaningful name
\end{itemize}
\end{itemize}

\begin{SQL}
SELECT DISTINCT f.DRINKER "Best Customers"
FROM FREQUENTS  f, SERVES  s
WHERE f.CAFE = s.CAFE AND s.COFFEE = 'Cold Brew'
\end{SQL}
\end{frame}


%***********************************************************
\begin{frame}[fragile]{JOIN}

\begin{SQL}
SELECT f.*
FROM FREQUENTS  AS f, SERVES  AS s
WHERE f.CAFE = s.CAFE AND s.COFFEE = 'Cold Brew'
\end{SQL}


\begin{itemize}
\item What kind of join is this? % theta
\begin{enumerate}[A]
\item Cartesian product / Cross join
\item Theta join
\item Natural join
\end{enumerate}

\end{itemize}
\end{frame}

%***********************************************************
\begin{frame}[fragile]{SELECT-FROM-WHERE}

\begin{SQL}
SELECT <attibute list>
FROM <tables>
WHERE <conditions>

SELECT f.* 
FROM FREQUENTS f
\end{SQL}


\begin{tabular}{|l|l|} \hline
\textbf{DRINKER} & \textbf{CAFE} \\ \hline
Chris & Double Trouble \\\hline
Chris & Tout Suite \\\hline
Risa & Java Lava \\ \hline
Risa & Double Trouble \\ \hline
\end{tabular}

\end{frame}

%***********************************************************
\begin{frame}[fragile]{SELECT Clauses}

\begin{tabular}{|l|l|l|} \hline
\textbf{Attribute} & \textbf{Example} & \textbf{Explanation} \\ \hline
Attibute list & $d$.lastName, $d$.firstName & Only the specified attributes \\\hline
* &* & All attributes from all relations \\\hline
<table name>.* & \FREQUENTS.* & All the attributes from relation \\
& &  <table name>\\ \hline
<alias name>.* & $f$.* & All the attributes from the relation \\
& & aliased to $f$\\ \hline
<math equation> & 1 + 3 & Evaluates the expression \\\hline
<constant> & \textquotesingle{CPA}\textquotesingle & Returns the specified constant \\
& 3 & \\ \hline
\end{tabular}

\end{frame}

%***********************************************************
\begin{frame}[fragile]{More SELECT Clauses}

\begin{tabular}{|l|l|l|} \hline
\textbf{Attribute} & \textbf{Example} & \textbf{Explanation} \\ \hline
<function> & \texttt{NOW}() & Current datetime \\
& \texttt{CONCAT}(<attribute and string constants) & Concatenates the values \\
& \texttt{COALESCE}(<attributes and constants>) & Returns the first non-NULL \\
&& argument \\ \hline
\texttt{DISTINCT} & & Eliminates duplicates \\ \hline
\end{tabular}

\end{frame}


%***********************************************************
\begin{frame}[fragile]{\texttt{CONCAT}}

DRINKER(FIRSTNAME, LASTNAME, DATEOFBIRTH)

\begin{SQL}
SELECT CONCAT(firstName, ' ', lastName) AS name
FROM DRINKER
\end{SQL}

\vspace{1em}

Also

\vspace{1em}

\begin{SQL}
SELECT firstName ||  ' ' || lastName AS name
FROM DRINKER
\end{SQL}

\end{frame}
%***********************************************************
\begin{frame}[fragile]{\texttt{COALESCE}}

	
\begin{itemize}
\item Returns the first non-NULL value in the list
\end{itemize}

\begin{SQL}
COALESCE(FIELDA, FIELDB, 'UNKNOWN')
\end{SQL}

\end{frame}

%***********************************************************
\begin{frame}[fragile]{\texttt{CASE}}

DRINKER(FIRSTNAME, LASTNAME, DATEOFBIRTH)

\begin{SQL}
SELECT CONCAT(firstName, ' ', lastName),
	CASE WHEN dateOfBirth IS NULL THEN 'Unknown' ELSE dateOfBirth::VARCHAR END  
	AS checkOver21
FROM Drinker
\end{SQL}

COURSE(CRN, COURSENAME, ROOMID)
\begin{SQL}
SELECT crn, 
	CASE crn
		WHEN 16671 THEN 'Grad Databases' 
   		WHEN 16670 THEN 'Undergrad Databases'
    		ELSE 'Unimportant' 
	END  AS "Course Type"
FROM Course
\end{SQL}

\end{frame}

%***********************************************************
\begin{frame}[fragile]{\texttt{FROM} Clause}

\begin{enumerate}
\item List Relation(s)/Table(s)/View(s)
\item Specify how they are related
\item Be explicit!
\begin{itemize}
\item (otherwise you get the Cartesian Product / Cross Join)
\end{itemize}
\end{enumerate}
\end{frame}
%%***********************************************************
%\begin{frame}{Join Visualization}
%
%{\centering \includegraphics[width=1\textwidth]{./lect06/joins3.pdf}}
%
%%{\centering \includegraphics[width=.7\textwidth]{./lect06/joinsSO.png}}
%%\url {https://stackoverflow.com/questions/406294/left-join-vs-left-outer-join-in-sql-server}
%\end{frame}
%
%***********************************************************
\begin{frame}{INNER JOIN}

R \texttt{INNER JOIN} S \texttt{ON} R.<att> = S.<att>

R \texttt{JOIN} S \texttt{ON} R.<att> = S.<att>

R  \texttt{NATURAL JOIN}  S

\begin{itemize}
\item Used to match up tuples from different relations
\item Includes only the relations with matching attribute values 
\end{itemize}

\end{frame}
%***********************************************************
\begin{frame}{Example: Inner Join}

COURSE (\underline{CRN}, NAME)

ENROLL (\underline{NETID}, \underline{CRN})

STUDENT (\underline{NETID})

\begin{columns}[T]
\begin{column}{0.3\textwidth}
STUDENT\\
\begin{tabular}{|l|l|} \hline
\textbf{NETID} & \textbf{NAME} \\ \hline
rbm2 & Risa \\\hline
abc1 & Andre \\\hline
bcd2 & Betty \\ \hline
cde4 & Chris \\ \hline
\end{tabular}\\
\end{column}
\begin{column}{0.3\textwidth}
ENROLL\\
\begin{tabular}{|l|l|} \hline
\textbf{NETID} & \textbf{CRN} \\ \hline
abc1 & 123 \\\hline
abc1 & 345\\\hline
cde4 & 123 \\ \hline
\end{tabular}
\end{column}
\begin{column}{0.3\textwidth}
COURSE\\
\begin{tabular}{|l|l|} \hline
\textbf{CRN} & \textbf{NAME} \\ \hline
123 & COMP 430 \\\hline
234 & COMP 533 \\\hline
345 & COMP 530 \\ \hline
\end{tabular}\\
\end{column}
\end{columns}


\begin{itemize}
\item[?] Which students have enrolled in a course?
\end{itemize}

\end{frame}
%***********************************************************
\begin{frame}[fragile]{Example: Inner Join}

COURSE (\underline{CRN}, NAME)

ENROLL (\underline{NETID}, \underline{CRN})


STUDENT (\underline{NETID}, \underline{NAME})


\begin{itemize}
\item Which students have enrolled in a course?
\item STUDENT $\bowtie_{NETID = NETID}$ ENROLL

\begin{SQL}
SELECT *
FROM STUDENT s INNER JOIN ENROLL e ON s.NETID = e.NETID
\end{SQL}

RESULTS\\
\begin{tabular}{|l|l|l|l|} \hline
\textbf{NETID} & \textbf{NAME} & \textbf{NETID} & \textbf{CRN}\\ \hline
abc1 & Andre & abc1  & 123 \\\hline
abc1 & Andre  & abc1 & 345 \\\hline
cde4 & Chris & cde4 & 123 \\ \hline
\end{tabular}\\
\item[?] How is a natural join different?
\end{itemize}
\end{frame}
%***********************************************************
\begin{frame}[fragile]{Example: Natural Join}

COURSE (\underline{CRN}, NAME)

ENROLL (\underline{NETID}, \underline{CRN})


STUDENT (\underline{NETID}, \underline{NAME})


\begin{itemize}
\item Which students have enrolled in a course?
\item STUDENT * ENROLL
\end{itemize}

\begin{SQL}
SELECT *
FROM STUDENT s NATURAL JOIN ENROLL e 
\end{SQL}

RESULTS\\
\begin{tabular}{|l|l|l|} \hline
\textbf{NETID} & \textbf{NAME} & \textbf{CRN}\\ \hline
abc1 & Andre   & 123 \\\hline
abc1 & Andre   & 345 \\\hline
cde4 & Chris  & 123 \\ \hline
\end{tabular}\\
\end{frame}
%***********************************************************
\begin{frame}{Left / Right Outer Join}

R \texttt{LEFT OUTER JOIN} S \texttt{ON} R.<att> = S.<att>

R \texttt{RIGHT OUTER JOIN} S \texttt{ON} R.<att> = S.<att>

\begin{itemize}
\item Used to match up tuples from different relations
\item Includes all the relations from the "outer" side 
\item If there is no matching tuple, assigns NULLs
\item Returns a relation with all the attributes of R $\bullet$ all the attributes of S
\item Tip: Pick one direction and use it consistently
\end{itemize}

\end{frame}

%***********************************************************
\begin{frame}{Example: Left Outer Join}

COURSE (\underline{CRN}, NAME)

ENROLL (\underline{NETID}, \underline{CRN})

STUDENT (\underline{NETID}, \underline{NAME})

\begin{columns}[T]
\begin{column}{0.3\textwidth}
STUDENT\\
\begin{tabular}{|l|l|} \hline
\textbf{NETID} & \textbf{NAME} \\ \hline
rbm2 & Risa \\\hline
abc1 & Andre \\\hline
bcd2 & Betty \\ \hline
cde4 & Chris \\ \hline
\end{tabular}\\
\end{column}
\begin{column}{0.3\textwidth}
ENROLL\\
\begin{tabular}{|l|l|} \hline
\textbf{NETID} & \textbf{CRN} \\ \hline
abc1 & 123 \\\hline
abc1 & 345\\\hline
cde4 & 123 \\ \hline
\end{tabular}
\end{column}
\begin{column}{0.3\textwidth}
COURSE\\
\begin{tabular}{|l|l|} \hline
\textbf{CRN} & \textbf{NAME} \\ \hline
123 & COMP 430 \\\hline
234 & COMP 533 \\\hline
345 & COMP 530 \\ \hline
\end{tabular}\\
\end{column}
\end{columns}


\begin{itemize}
\item[?] Which students haven't enrolled in any courses?
\end{itemize}

\end{frame}

%***********************************************************
\begin{frame}[fragile]{Example: Left Outer Join}

COURSE (\underline{CRN}, NAME)

ENROLL (\underline{CRN}, \underline{NETID})

STUDENT (\underline{NETID}, \underline{NAME})


\begin{itemize}
\item Which students haven't enrolled in any courses?
\item STUDENT $\leftouterjoin_{NETID = NETID}$ ENROLL

\begin{SQL}
SELECT *
FROM STUDENT s LEFT OUTER JOIN ENROLL e ON s.NETID = e.NETID
WHERE e.CRN IS NULL
\end{SQL}
\end{itemize}

RESULTS\\
\begin{tabular}{|l|l|l|l|} \hline
\textbf{NETID} & \textbf{NAME} & \textbf{NETID} & \textbf{CRN} \\ \hline
rbm2 & Risa & NULL & NULL\\\hline
bcd2 & Betty & NULL & NULL \\ \hline
\end{tabular}\\


\end{frame}


%***********************************************************
\begin{frame}[fragile]{Example: Right Outer Join}

COURSE (\underline{CRN}, NAME)

ENROLL (\underline{NETID}, \underline{CRN})

STUDENT (\underline{NETID}, \underline{NAME})


\begin{itemize}
\item[?] What question does this query answer?
\item ENROLL $\rightouterjoin_{NETID = NETID}$ COURSE

\begin{SQL}
SELECT *
FROM ENROLL e RIGHT OUTER JOIN COURSE c ON e.CRN = c.CRN
WHERE e.CRN IS NULL
\end{SQL}
\end{itemize}


\end{frame}
%***********************************************************
\begin{frame}[fragile]{Example: Right Outer Join}

COURSE (\underline{CRN}, NAME)

ENROLL (\underline{NETID}, \underline{CRN})

STUDENT (\underline{NETID}, \underline{NAME})

\begin{itemize}
\item What question does this query answer?
\item ENROLL $\rightouterjoin_{NETID = NETID}$ COURSE
% which classes have no students?
\begin{SQL}
SELECT *
FROM ENROLL e RIGHT OUTER JOIN COURSE c ON e.CRN = c.CRN
WHERE e.CRN IS NULL
\end{SQL}
\end{itemize}

RESULTS\\
\begin{tabular}{|l|l|l|l|} \hline
\textbf{NETID} & \textbf{CRN} & \textbf{CRN} & \textbf{NAME}\\ \hline
NULL & NULL  & 234 & COMP 533\\\hline
\end{tabular}\\


\end{frame}

%***********************************************************
\begin{frame}{Full Outer Join}

R \texttt{FULL OUTER JOIN} S \texttt{ON} R.<att> = S.<att>

R $\fullouterjoin_{R.<att> = S.<att>}$ S

\begin{itemize}
\item Used to match up tuples from different relations
\item Includes all the relations from both sides 
\item If there is no matching tuple, assigns NULLs
\item Returns a relation with all the attributes of R $\bullet$ all the attributes of S
\end{itemize}
\end{frame}

%***********************************************************
\begin{frame}[fragile]{Example: Full Outer Join}

STUDENT (\underline{NETID}, NAME)

TEAM (\underline{TEAMNAME}, CAPTAINNETID)


\begin{columns}[T]
\begin{column}{0.4\textwidth}
STUDENT\\
\begin{tabular}{|l|l|} \hline
\textbf{NETID} & NAME  \\ \hline
ghi8 & Gary  \\\hline
hij2 & Holly \\\hline
 ijk12 & Isabel  \\ \hline
\end{tabular}\\
\end{column}
\begin{column}{0.4\textwidth}

TEAM\\
\begin{tabular}{|l|l|} \hline
\textbf{TEAMNAME} & CAPTAINNETID  \\ \hline
Peanut butter & ghi8  \\\hline
Jelly & NULL \\\hline
\end{tabular}\\
\end{column}
\end{columns}

\begin{SQL}
SELECT s.NAME, t.TEAMNAME
FROM STUDENT s FULL OUTER JOIN TEAM t ON s.NETID = t.CAPTAINNETID
\end{SQL}
\begin{itemize}
\item[?] What does this expression represent?
% students who are team captains and teams  without captains
\item[?] How might it be useful?
\end{itemize}


%create table team(
% teamname varchar(20),
% CAPTAINNETID varchar(10)
%);
%
%insert into team values
%('Peanut butter', 'ghi8'),
%('Jelly', NULL);
%
%insert into student values
%('ghi8', 'Gary'),
%('hij2', 'Holly'),
%('ijk12', 'Isabel');
\end{frame}


%***********************************************************
\begin{frame}[fragile]{Example: Full Outer Join}

STUDENT (\underline{NETID}, NAME)

TEAM (\underline{TEAMNAME}, CAPTAINNETID)

\begin{SQL}
SELECT s.NAME, t.TEAMNAME
FROM STUDENT s FULL OUTER JOIN TEAM t ON s.NETID = t.NETID
\end{SQL}

RESULT\\
\begin{tabular}{|l|l|} \hline
\textbf{NAME} & \textbf{NAME}  \\ \hline
Gary & Peanut butter \\\hline
NULL & Jelly \\\hline
 Holly & NULL  \\ \hline
 Isabel & NULL  \\ \hline
\end{tabular}\\

\end{frame}

%***********************************************************
\begin{frame}[fragile]{Self Join}

\begin{SQL}
R AS R1 JOIN R AS R2 ON R1.<att> = R2.<att>
\end{SQL}

\begin{itemize}
\item Used to match up tuples from relation R back to itself
\item Any type of JOIN may be used 
\item Returns a relation with all the attributes of R $\bullet$ all the attributes of R
\end{itemize}
\end{frame}

%***********************************************************
\begin{frame}[fragile]{Example: Self Join}

FACULTY (\underline{NETID}, NAME, MGRNETID )

FACULTY\\
\begin{tabular}{|l|l|l|} \hline
\textbf{NETID} & \textbf{NAME} & \textbf{MGRNETID} \\ \hline
rbm2 & Risa & abc1 \\\hline
abc1 & Andre & bcd2 \\\hline
bcd2 & Betty & abc1  \\ \hline
cde4 & Chris & NULL\\ \hline
\end{tabular}\\

%create table faculty(
% netid varchar(10),
% name varchar(10),
% mgrnetid varchar(10)
%)

%insert into faculty values
%	('rbm2', 'risa', 'bcd2'),
%	('abc1', 'Andre', 'bcd2'),
%	('bcd2', 'betty', 'cde4'),
%	('cde4', 'Chris', NULL);

\begin{SQL}
SELECT f.NAME, Mgr.Name
FROM FACULTY f JOIN FACULTY Mgr ON f.MGRNETID = Mgr.NETID
\end{SQL}

\begin{itemize}
\item[?] What does this expression represent?
\begin{enumerate}[A]
\item Every faculty member paired with every manager
\item Every faculty member who has. a manager, paired with that manager % this one
\end{enumerate}
\end{itemize}

\end{frame}

%***********************************************************
\begin{frame}[fragile]{Example: Self Join}

FACULTY (\underline{NETID}, NAME, MGRNETID )

FACULTY\\
\begin{tabular}{|l|l|l|} \hline
\textbf{NETID} & \textbf{NAME} & \textbf{MGRNETID} \\ \hline
rbm2 & Risa & bcd2 \\\hline
abc1 & Andre & bcd2 \\\hline
bcd2 & Betty & cde4  \\ \hline
cde4 & Chris & NULL\\ \hline
\end{tabular}\\


\begin{SQL}
SELECT f.NAME, Mgr.Name
FROM FACULTY f JOIN FACULTY Mgr ON f.MGRNETID = Mgr.NETID
\end{SQL}

RESULT\\
\begin{tabular}{|l|l|} \hline
\textbf{NAME} & \textbf{NAME}  \\ \hline
Risa & Betty \\\hline
Andre & Betty \\\hline
 Betty & Chris  \\ \hline
\end{tabular}\\

\end{frame}

%***********************************************************
\begin{frame}{True / False Questions}

\begin{enumerate}
\item SQL queries typically start with a SELECT clause %T
\item SELECT clauses must contain at least one attribute %F - could be a constant or function
\item Every LEFT JOIN expression can be written as a RIGHT JOIN expression %T
\item A Cartesian Product is also known as a CROSS JOIN or CROSS PRODUCT % T
\item SELF JOINs run the query twice %F
\item SELF JOINs use the same table more than once % T
\item SQL statements ignore whitespace %T
\item Attributes can be renamed to "user friendly" labels %T
\end{enumerate}
\end{frame}
	
%***********************************************************

\begin{frame}{Wrap up}
\begin{itemize}
	\item[?] How can we use what we learned today?
	\vspace{2em}
	\item[?] What do we know now that we didn't know before?
\end{itemize}


\end{frame}

\end{document}
