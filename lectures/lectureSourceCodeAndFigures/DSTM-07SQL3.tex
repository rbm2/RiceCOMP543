%Copyright 2019 Christopher M. Jermaine (cmj4@rice.edu) and Risa B. Myers (rbm2@rice.edu)
%
%Licensed under the Apache License, Version 2.0 (the "License");
%you may not use this file except in compliance with the License.
%You may obtain a copy of the License at
%
%    https://www.apache.org/licenses/LICENSE-2.0
%
%Unless required by applicable law or agreed to in writing, software
%distributed under the License is distributed on an "AS IS" BASIS,
%WITHOUT WARRANTIES OR CONDITIONS OF ANY KIND, either express or implied.
%See the License for the specific language governing permissions and
%limitations under the License.
%===============================================================%
\documentclass[aspectratio=169]{beamer}
\mode<presentation> 
{
\usetheme[noshadow, minimal,numbers,riceb,nonav]{Rice}
\usefonttheme[onlymath]{serif}
\setbeamercovered{transparent}
}
\useinnertheme{rectangles}
\usepackage{colortbl}

\usepackage[english]{babel}

\usepackage{mathptmx}
\usepackage{helvet}
\usepackage{courier}
\usepackage[T1]{fontenc}
\usepackage{trajan}
\usepackage{ textcomp }
\usepackage{amssymb}

%https://tex.stackexchange.com/questions/20740/symbols-for-outer-joins
\def\ojoin{\setbox0=\hbox{$\bowtie$}%
  \rule[-.02ex]{.25em}{.4pt}\llap{\rule[\ht0]{.25em}{.4pt}}}
\def\leftouterjoin{\mathbin{\ojoin\mkern-5.8mu\bowtie}}
\def\rightouterjoin{\mathbin{\bowtie\mkern-5.8mu\ojoin}}
\def\fullouterjoin{\mathbin{\ojoin\mkern-5.8mu\bowtie\mkern-5.8mu\ojoin}}


\usepackage{listings}

\newenvironment{noindentitemize}
{ \begin{itemize}
 \setlength{\itemsep}{1.5ex}
  \setlength{\parsep}{0pt}   
  \setlength{\parskip}{0pt}
 \addtolength{\leftskip}{-2em}
 }
{ \end{itemize} }

\newenvironment{noindentitemize2}
{ \begin{itemize}
  \setlength{\itemsep}{0ex}
  \setlength{\parskip}{0pt}
  \setlength{\parsep}{0pt}   
  \addtolength{\leftskip}{-2em}  }
{ \end{itemize} }

\lstnewenvironment{SQL}
  {\lstset{
        aboveskip=5pt,
        belowskip=5pt,
        escapechar=!,
        mathescape=true,
        upquote=true,
        language=SQL,
        basicstyle=\linespread{0.94}\ttfamily\footnotesize,
        morekeywords={FOR, EACH, WITH, PARTITION, AND, ALL, TEST, WHETHER, PROBABILITY},
        deletekeywords={VALUE, PRIOR},
        showstringspaces=true}
        \vspace{0pt}%
        \noindent\minipage{0.47\textwidth}}
  {\endminipage\vspace{0pt}}


\newcommand{\NULL}{\texttt{NULL}} 
\newcommand{\WHERE}{\texttt{WHERE}} 
\newcommand{\ALL}{\texttt{ALL}} 
\newcommand{\UNION}{\texttt{UNION}} 
\newcommand{\EXCEPT}{\texttt{EXCEPT}} 
\newcommand{\LIKES}{\textrm{LIKES}} 
\newcommand{\FREQUENTS}{\textrm{FREQUENTS}} 
\newcommand{\SERVES}{\textrm{SERVES}} 
\newcommand{\CAFE}{\textrm{CAFE}} 
\newcommand{\COFFEE}{\textrm{COFFEE}} 
\newcommand{\DRINKER}{\textrm{DRINKER}} 
\newcommand{\CB}{\textrm{\textquotesingle{Cold} Brew\textquotesingle}} 
\newcommand{\CBGOOD}{\textrm{CBGOOD}} 
\newcommand{\ALLPEEPS}{\textrm{ALLPEEPS}} 
\newcommand{\ALLCOMBOS}{\textrm{ALLCOMBOS}} 
\newcommand{\NOGOODCOFFEE}{\textrm{NOGOODCOFFEE}} 

\setbeamerfont{block body}{size=\tiny}

%===============================================================%

\title[]
{Tools \& Models for Data Science}

\subtitle{SQL Aggregations and Grouping}

\author[]{Risa Myers}
\institute
{
  Rice University
}

\date[]{}


\begin{document}

\begin{frame}
 \titlepage
\end{frame}

%***********************************************************
\begin{frame}{Aggregations}

\begin{itemize}
\item Can compute simple statistics using built-in SQL functions
	\begin{itemize}
	\item \texttt{SUM}
	\item \texttt{AVG}
	\item \texttt{COUNT}
	\item \texttt{MAX}
	\item \texttt{MIN}
	\item etc.
	\end{itemize}
\item[?] What do all of these aggregates have in common?
\end{itemize}
\end{frame}

%***********************************************************
\begin{frame}{Our First Aggregation}

RATES (DRINKER, COFFEE, SCORE)

\begin{itemize}
\item[?]  What is the average coffee rating given by Risa?
\end{itemize}
\end{frame}

%***********************************************************
\begin{frame}[fragile]{Our First Aggregation}

RATES (DRINKER, COFFEE, SCORE)

\begin{itemize}
\item What is the average coffee rating given by Risa?
\end{itemize}

\begin{SQL}
SELECT AVG (r.SCORE)
FROM RATES r
WHERE r.DRINKER = 'Risa'
\end{SQL}
\end{frame}

%***********************************************************

\begin{frame}{\texttt{COUNT DISTINCT}}


\begin{itemize}
\item[] RATES (DRINKER, COFFEE, SCORE)
\item[?] How many coffees has Risa rated?
\item Note: RATES does not have a primary key
\item[?] What are the reperecussions? % there can be duplicate tuples
\end{itemize}
\end{frame}

%***********************************************************
\begin{frame}[fragile]{\texttt{COUNT DISTINCT}}

RATES (DRINKER, COFFEE, SCORE)

\begin{itemize}
\item How many coffees has Risa rated?
\item[?] Does this work?
\end{itemize}

\begin{SQL}
SELECT COUNT (*)
FROM RATES r
WHERE r.DRINKER = 'Risa'
\end{SQL}

\end{frame}

%***********************************************************
\begin{frame}[fragile]{\texttt{COUNT DISTINCT}}

RATES (DRINKER, COFFEE, SCORE)

\begin{itemize}
\item How many coffees has Risa rated?
\item Does this work?
\end{itemize}

\begin{SQL}
SELECT COUNT (*)
FROM RATES r
WHERE r.DRINKER = 'Risa'
\end{SQL}

\begin{itemize}
	\item Counts the number of ratings due to Risa.
	\item[?] Count the number of different types of coffee drinks that Risa has rated
\end{itemize}
\end{frame}

%***********************************************************
\begin{frame}[fragile]{\texttt{COUNT DISTINCT}}

RATES (DRINKER, COFFEE, SCORE)

\begin{itemize}
\item How many coffees has Risa rated?
\item This gives us the actual number rated:
\end{itemize}

\begin{SQL}
SELECT COUNT (DISTINCT r.COFFEE)
FROM RATES r
WHERE r.DRINKER = 'Risa'
\end{SQL}
\end{frame}

%***********************************************************
\begin{frame}{\texttt{GROUP BY}}

RATES (DRINKER, COFFEE, SCORE)

\begin{itemize}
\item It is often desirable to compute an aggregate at a finer level of granularity.
\item[?] What is the average rating for each coffee?
\end{itemize}
\end{frame}

%***********************************************************
\begin{frame}[fragile]{\texttt{GROUP BY}}

RATES (DRINKER, COFFEE, SCORE)

\begin{itemize}
\item It is often desirable to compute an aggregate at a finer level of granularity.
\item What is the average rating for each coffee?
\end{itemize}

\begin{SQL}
SELECT r.COFFEE, AVG (r.SCORE)
FROM RATES r
GROUP BY r.COFFEE
\end{SQL}

\begin{itemize}
	\item This first groups the relation into subgroups
	\item Every tuple in the subgroup has the same value for r.COFFEE
	\item Then the aggregate runs over each subgroup independently
\end{itemize}	
\end{frame}

%***********************************************************
\begin{frame}[fragile]{\texttt{GROUP BY}}

\begin{SQL}
SELECT r.COFFEE, AVG (r.SCORE)
FROM RATES r
GROUP BY r.COFFEE
\end{SQL}

\begin{noindentitemize}
	\item Example input: 
\end{noindentitemize}	

\begin{SQL}
('Chris', 'Cold Brew', 1)
('Chris', 'Turkish Coffee', 5)
('Jorge', 'Cold Brew', 1)
('Jorge', 'Chai Latte', 3)
('Risa', 'Cold Brew', 4)
('Risa', 'Cold Brew', 5)
('Risa', 'Espresso', 2)
\end{SQL}

\begin{noindentitemize}
	\item[?]  What is the output? 
\end{noindentitemize}	

\end{frame}

%***********************************************************
\begin{frame}[fragile]{\texttt{GROUP BY}}

\begin{SQL}
SELECT r.COFFEE, AVG (r.SCORE)
FROM RATES r
GROUP BY r.COFFEE
\end{SQL}

\begin{noindentitemize}
	\item Example input: 
\end{noindentitemize}	

\begin{SQL}
('Chris', 'Cold Brew', 1)
('Chris', 'Turkish Coffee', 5)
('Jorge', 'Cold Brew', 1)
('Jorge', 'Chai Latte', 3)
('Risa', 'Cold Brew', 4)
('Risa', 'Cold Brew', 5)
('Risa', 'Espresso', 2)
\end{SQL}

\begin{noindentitemize}
	\item  What is the output? 
\end{noindentitemize}	

\begin{SQL}
('Turkish Coffee', 5)
('Chai Latte', 3)
('Cold Brew', 2.75)
('Espresso', 2)
\end{SQL}

\begin{noindentitemize}
	\item Take care with integer arithmetic! 
\end{noindentitemize}	
\end{frame}

%***********************************************************
\begin{frame}[fragile]{\texttt{GROUP BY}}

\begin{SQL}
SELECT r.COFFEE, AVG (R.SCORE)
FROM RATES r
GROUP BY r.COFFEE
\end{SQL}

\begin{itemize}
	\item Note: If you have an attribute outside of an aggregate function in an aggregate query
	\item Example: r.COFFEE here
	\item Then you must have grouped by that attribute
	\item Or the query will not compile
	\item[?] Why?
\end{itemize}	

\end{frame}

%***********************************************************
\begin{frame}{\texttt{GROUP BY} Conceptually}
\footnotesize{
\begin{itemize}
\item Given the following data
\begin{tabular}{|l|c|c| }  \hline
\textrm{DRINKER} & \textrm{COFFEE} & \textrm{SCORE}\\ \hline
Risa & Espresso & 2\\ \hline
Chris & Cold Brew & 1\\ \hline
Chris & Turkish Coffee & 5 \\ \hline
Risa & Cold Brew & 4 \\ \hline
Risa & Cold Brew & 5 \\ \hline
\end{tabular}
\item[?] What is each drinker's average coffee rating?
\end{itemize}

}
\end{frame}

\begin{frame}{\texttt{GROUP BY} Conceptually}
\footnotesize{
\begin{itemize}
\item Given the following data
\begin{tabular}{|l|c|c| }  \hline
\textrm{DRINKER} & \textrm{COFFEE} & \textrm{SCORE}\\ \hline
Risa & Espresso & 2\\ \hline
Chris & Cold Brew & 1\\ \hline
Chris & Turkish Coffee & 5 \\ \hline
Risa & Cold Brew & 4 \\ \hline
Risa & Cold Brew & 5 \\ \hline
\end{tabular}
\item[?] What is each drinker's average coffee rating?
\end{itemize}
\begin{enumerate}
\item \texttt{GROUP BY} DRINKER
\begin{tabular}{|l|c|c| }  \hline
\textrm{DRINKER} & \textrm{COFFEE} & \textrm{SCORE}\\ \hline
Chris & Cold Brew & 1\\ \hline
Chris & Turkish Coffee & 5 \\ \hline \hline
Risa & Espresso & 2\\ \hline
Risa & Cold Brew & 4 \\ \hline
Risa & Cold Brew & 5 \\ \hline
\end{tabular}
\item Aggregate
\begin{tabular}{|l|c| }  \hline
\textrm{DRINKER} & \textrm{AVGSCORE}\\ \hline
Chris & 3\\ \hline
Risa & 3.67 \\ \hline
\end{tabular}
\end{enumerate}
}
\end{frame}

%***********************************************************
\begin{frame}[fragile]{\texttt{HAVING}}

RATES (DRINKER, COFFEE, SCORE)
\begin{noindentitemize}
\item[?] What is the highest rated type of coffee, on average, considering only coffees that have at least 3 ratings?\\ 
\item From last class:
\end{noindentitemize}

\begin{SQL}
CREATE VIEW COFFEE_AVG_RATING AS
   SELECT r.COFFEE, AVG (r.SCORE) AS AVG_RATING
   FROM RATES r
   GROUP BY r.COFFEE
\end{SQL}

\begin{SQL}
SELECT a.COFFEE
FROM COFFEE_AVG_RATING a
WHERE a.AVG_RATING = (SELECT MAX(a.AVG_RATING)
                      FROM COFFEE_AVG_RATING a)
\end{SQL}
\begin{noindentitemize}
\item[?] How do we check for at least 3 ratings?
\end{noindentitemize}

\end{frame}

%***********************************************************
\begin{frame}[fragile]{\texttt{HAVING}}

RATES (DRINKER, COFFEE, SCORE)

\begin{noindentitemize}
\item What is the highest rated type of coffee, on average, considering only coffees that have at least 3 ratings?\\ 
\end{noindentitemize}
\begin{noindentitemize}
\item Change COFFEE\_AVG\_RATING to:
\end{noindentitemize}

\begin{SQL}
CREATE VIEW COFFEE_AVG_RATING AS
   SELECT r.COFFEE, AVG(r.SCORE) AS AVG_RATING
   FROM RATES r
   GROUP BY COFFEE
   HAVING COUNT(*) >= 3
\end{SQL}
\end{frame}
% added the HAVING CLAUSE

%***********************************************************
\begin{frame}{\texttt{HAVING} Conceptually}
\scriptsize{
\begin{noindentitemize}
\item Given the following data
\begin{tabular}{|l|c|c| }  \hline
\textrm{DRINKER} & \textrm{COFFEE} & \textrm{SCORE}\\ \hline
Risa & Espresso & 2\\ \hline
Chris & Cold Brew & 1\\ \hline
Chris & Turkish Coffee & 5 \\ \hline
Risa & Cold Brew & 4 \\ \hline
Risa & Cold Brew & 5 \\ \hline
\end{tabular}
\item[?] What is the highest rated type of coffee, on average, considering only coffees that have at least 3 ratings?\\
\end{noindentitemize}
\begin{enumerate}
\item \texttt{GROUP BY} COFFEE
\begin{tabular}{|l|c|c| }  \hline
\textrm{DRINKER} & \textrm{COFFEE} & \textrm{SCORE}\\ \hline
Chris & Cold Brew & 1\\ \hline
Risa & Cold Brew & 4 \\ \hline
Risa & Cold Brew & 5 \\ \hline \hline
Chris & Turkish Coffee & 5 \\ \hline \hline
Risa & Espresso & 2\\ \hline
\end{tabular}
\item Aggregate
\begin{tabular}{|l|c| }  \hline
\textrm{COFFEE} & \textrm{AVGSCORE}\\ \hline 
Cold Brew & 3.33\\ \hline\hline
 Turkish Coffee & 5 \\ \hline \hline
 Espresso & 2 \\ \hline 
\end{tabular}
\item \texttt{HAVING} COUNT(*) >=  3
\begin{tabular}{|l|c| }  \hline
\textrm{DRINKER} & \textrm{AVGSCORE}\\ \hline
Cold Brew & 3.33\\ \hline
\end{tabular}

\end{enumerate}

}
\end{frame}

%***********************************************************
%\begin{frame}{HAVING}
%
%RATES (DRINKER, COFFEE, SCORE)
%
%\begin{itemize}
%\item Example: What is the highest rated coffee, on average, considering only coffees that have at least 10 ratings?
%\end{itemize}

%***********************************************************
\begin{frame}{Aggregate Functions  \& NULL}

What about NULL?


\begin{itemize}
\item \texttt{COUNT}(1) or \texttt{COUNT}(*) will count every row
\item \texttt{COUNT}(<attribute>) will count NON-NULL values
\item \texttt{AVG, MIN, MAX}, etc. ignore \NULL\ values
\item \texttt{GROUP BY} includes a row for \NULL
\end{itemize}
\end{frame}



%***********************************************************
\begin{frame}{Subquery in \texttt{FROM} Revisited}

RATES (DRINKER, COFFEE, SCORE)

\begin{itemize}
\item Can have a subquery in \texttt{FROM} clause, treated as a temporary table
\item[?] What is the highest rated coffee, on average?
\end{itemize}
\end{frame}

%***********************************************************
\begin{frame}[fragile]{Subquery in \texttt{FROM} Revisited}

RATES (DRINKER, COFFEE, SCORE)

\begin{itemize}
\item Can have a subquery in \texttt{FROM} clause, treat as a temporary table
\item What is the highest rated coffee, on average?
\end{itemize}

\begin{SQL}
SELECT a.COFFEE
FROM (SELECT r.COFFEE, AVG (r.SCORE) AS AVG_RATING
      FROM RATES r
      GROUP BY r.COFFEE) a
WHERE a.AVG_RATING = (SELECT MAX(a.AVG_RATING)
                      FROM (SELECT r.COFFEE, AVG (r.SCORE) 
                               AS AVG_RATING
                            FROM RATES r
                            GROUP BY r.COFFEE) a)
\end{SQL}
\end{frame}

%***********************************************************
\begin{frame}[fragile]{Subquery in \texttt{FROM} Revisited}

RATES (DRINKER, COFFEE, SCORE)

\begin{noindentitemize}
\item Note: The code is a lot cleaner with a view!
\end{noindentitemize}

\begin{SQL}
CREATE VIEW COFFEE_AVG_RATING AS
SELECT r.COFFEE, AVG (r.SCORE) AS AVG_RATING
FROM RATES r
GROUP BY r.COFFEE
\end{SQL}

\begin{SQL}
SELECT a.COFFEE
FROM COFFEE_AVG_RATING a
WHERE a.AVG_RATING = (SELECT MAX(a.AVG_RATING)
                      FROM COFFEE_AVG_RATING a)
\end{SQL}
\end{frame}


%***********************************************************
\begin{frame}[fragile]{\texttt{TOP} k / \texttt{LIMIT} k}

RATES (DRINKER, COFFEE, SCORE)

\begin{itemize}
\item What is the highest rated coffee, on average?
\item Actually, can be a lot easier with \texttt{LIMIT} k.
\end{itemize}

\begin{SQL}
CREATE VIEW COFFEE_AVG_RATING AS
SELECT r.COFFEE, AVG (r.SCORE) AS AVG_RATING
FROM RATES r
GROUP BY r.COFFEE
\end{SQL}

\begin{SQL}
SELECT  a.COFFEE
FROM COFFEE_AVG_RATING a
ORDER BY a.AVG_RATING DESC LIMIT 1;
\end{SQL}
\end{frame}

%***********************************************************
\begin{frame}[fragile]{\texttt{TOP} k / \texttt{LIMIT} k}

\begin{itemize}
\item What is the highest rated coffee, on average?
\item Actually, can be a lot easier with \texttt{LIMIT} k.
\end{itemize}

\begin{itemize}
	\item Can choose \texttt{ASC} or \texttt{DESC}
	\item Finally: note that \texttt{ORDER BY} can be used without \texttt{LIMIT}
	\item[]
	\item[?] Will this approach always work?
\end{itemize}
\end{frame}
%***********************************************************
\begin{frame}{More True/False Questions}

\begin{enumerate}
\item ORDER BY only sorts by a single attribute %F 
\item All attributes in the ORDER BY clause are sorted by the same ASC or DESC rule %F
\item GROUP BY ignores NULL values %F
\item Aggregate functions ignore NULL values  %T
\item Aggregate functions are a pain to use and are slow. You are better off implementing your own version of them %F

\end{enumerate}

\end{frame}

%***********************************************************
\begin{frame}{Questions?}
\end{frame}
\end{document}
