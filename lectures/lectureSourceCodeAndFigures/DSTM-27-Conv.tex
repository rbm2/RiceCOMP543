%Copyright 2019 Christopher M. Jermaine (cmj4@rice.edu) and Risa B. Myers (rbm2@rice.edu)
%
%Licensed under the Apache License, Version 2.0 (the "License");
%you may not use this file except in compliance with the License.
%You may obtain a copy of the License at
%
%    https://www.apache.org/licenses/LICENSE-2.0
%
%Unless required by applicable law or agreed to in writing, software
%distributed under the License is distributed on an "AS IS" BASIS,
%WITHOUT WARRANTIES OR CONDITIONS OF ANY KIND, either express or implied.
%See the License for the specific language governing permissions and
%limitations under the License.
%===============================================================
\documentclass[aspectratio=169,14pt]{beamer}

\AtBeginSection[]
{
  \begin{frame}
    \frametitle{Table of Contents}
    \tableofcontents[currentsection]
  \end{frame}
}

\mode<presentation>
{
\usetheme[noshadow,minimal,numbers,nonav]{Rice}
\usefonttheme[onlymath]{serif}
% \setbeamercovered{transparent}
}

\useinnertheme{rectangles}

\usepackage[english]{babel}

\usepackage{amsmath}
\usepackage{mathptmx}
\usepackage{helvet}
\usepackage{courier}
\usepackage[T1]{fontenc}
\usepackage{trajan}
\usepackage{ textcomp }
\usepackage{listings}

\newenvironment{noindentitemize}
{ \begin{itemize}
 \setlength{\itemsep}{1.5ex}
  \setlength{\parsep}{0pt}
  \setlength{\parskip}{0pt}
 \addtolength{\leftskip}{-2em}
 }
{ \end{itemize} }

\newenvironment{noindentitemize2}
{ \begin{itemize}
  \setlength{\itemsep}{0ex}
  \setlength{\parskip}{0pt}
  \setlength{\parsep}{0pt}
  \addtolength{\leftskip}{-2em}  }
{ \end{itemize} }

\lstnewenvironment{SQL}
  {\lstset{
        aboveskip=5pt,
        belowskip=5pt,
        escapechar=!,
        mathescape=true,
        upquote=true,
        language=SQL,
        basicstyle=\linespread{0.94}\ttfamily\footnotesize,
        morekeywords={WHILE, DO, END},
        deletekeywords={VALUE, PRIOR},
        showstringspaces=true}
        \vspace{0pt}%
        \noindent\minipage{0.47\textwidth}}
  {\endminipage\vspace{0pt}}

\newcommand{\LIKES}{\textrm{LIKES}}
\newcommand{\FREQUENTS}{\textrm{FREQUENTS}}
\newcommand{\SERVES}{\textrm{SERVES}}
\newcommand{\CAFE}{\textrm{CAFE}}
\newcommand{\COFFEE}{\textrm{COFFEE}}
\newcommand{\DRINKER}{\textrm{DRINKER}}
\newcommand{\ALLPEEPS}{\textrm{ALLPEEPS}}
\newcommand{\ALLCOMBOS}{\textrm{ALLCOMBOS}}


%############################################


% TikZ
\usepackage{tikz}
\usepackage{pgflibraryshapes}
\usetikzlibrary{decorations.text}
\usetikzlibrary{decorations.pathreplacing}
\usetikzlibrary{backgrounds}
\usetikzlibrary{calc}
\usetikzlibrary{arrows}
\usetikzlibrary{snakes}
\usetikzlibrary{tikzmark,positioning}
\usetikzlibrary{shapes}
\usetikzlibrary{shapes.misc}
\usetikzlibrary{mindmap}
\usetikzlibrary{positioning}
\usetikzlibrary{math}
\usetikzlibrary{calc}
\usetikzlibrary{patterns}
%%\input{figures/tikzColors}
\definecolor{cffffff}{RGB}{255,255,255}
\definecolor{cff00c0}{RGB}{255,0,192}
\definecolor{c0f00ff}{RGB}{15,0,255}
\definecolor{c00ff20}{RGB}{0,255,32}
\definecolor{cfffd00}{RGB}{255,253,0}
\definecolor{c00e0ff}{RGB}{0,224,255}
\usepackage{pgfplots}
\usepackage{pgfplotstable}
% \usepackage{adjustbox}
\usepackage[export]{adjustbox}

\usepackage{multirow}
\usepackage{multimedia}
\usepackage{bigints}
\usepackage{varwidth}
\usepackage[none]{hyphenat}
\usepackage{fontawesome}



%############################################




\setbeamerfont{block body}{size=\tiny}

%===============================================================%

\title[]
{Tools \& Models for Data Science}

\subtitle{Neural Networks: Bells and Whistles}

\author[]{Jonas Actor and Risa Myers}
\institute
{  Rice University  }

\date[]{}

% \subject{Beamer}

\begin{document}

\begin{frame}
 \titlepage
\end{frame}



%***********************************************************
\begin{frame}
    \frametitle{Motivation}

NN are large nonconvex optimization problems
~\\~\\
\quad \dots large nonconvex optimization problems are hard to solve

\end{frame}

\begin{frame}
\frametitle{Objective}
Learn common ways to $\rule{3cm}{0.15mm}$ of NNs:
\begin{enumerate}
\item combat model complexity
\item improve generalization
\item overcome difficulties in training
\end{enumerate}

\end{frame}


%***********************************************************


\begin{frame}
\frametitle{Bells and Whistles}
\tableofcontents
\end{frame}

\section{Convolutional Neural Networks}
\begin{frame}
\frametitle{What is a discrete convolution?}
 Linear operator that applies a small kernel everywhere
 \begin{itemize}
     \item local point-wise multiplication
     \item reduce via sum
 \end{itemize}
\end{frame}


\begin{frame}
    \frametitle{What is a discrete convolution?}
    $k$: kernel of size $n_k$ a small number \\
    $x$: vector ~\\
$$ (x * k)_i = \sum_{j=-n_k}^{n_k} x_{i+j} \, k_j $$
\end{frame}

\begin{frame}
\frametitle{Example of 1D Convolution}
\begin{equation*} \begin{split}
x &= \begin{bmatrix} 1 & 2 & -3 & 0 & 4 & -1 & 2 \end{bmatrix} \\
k &= \begin{bmatrix} -1 & 0 & 1 \end{bmatrix} \\
\end{split} \end{equation*}
\end{frame}


\begin{frame}
    \frametitle{Example of 1D Convolution}
    \begin{center}
\begin{tikzpicture}
    \onslide<1->{
\foreach \s/\x in {0/[, 1/1, 2/2, 3/-3, 4/0, 5/-4, 6/-1, 7/2, 8/]}
{
    \node[xshift=\s*25] (0,0) {$\x$};
}
}
\node[xshift=0,yshift=-100] (0,0) {$[$};
\node[xshift=200,yshift=-100] (0,0) {$]$};
\foreach \ss/\v in {2/-4,3/-2,4/-1,5/-1,6/6}
{
    \onslide<\ss-\ss>{
        \node[xshift=(\ss-1)*25,yshift=-25] (0,0) {$-1$};
        \node[xshift=\ss*25,yshift=-25] (0,0) {$0$};
        \node[xshift=(\ss+1)*25,yshift=-25] (0,0) {$1$};
        \draw[|-|,below,xshift=(\ss-1)*25,yshift=-40] (0,0) -- (2,0) ;
        \draw[->,xshift=(\ss-1)*25,yshift=-40] (1,0) -- (1,-1) ;
    }
    \onslide<\ss->{ \node[xshift=\ss*25,yshift=-100] (0,0) {$\v$}; }

}
\end{tikzpicture}
\end{center}
\end{frame}



\begin{frame}
    \frametitle{What to do at the boundary}
\begin{enumerate}
    \item ignore them! (`valid')
    \item pad with zeros (`same')
    \item wrap around the end (`periodic')
\end{enumerate}
\end{frame}



\begin{frame}
    \frametitle{Example with Valid Padding}
    \begin{center}
\begin{tikzpicture}
    \onslide<1->{
\foreach \s/\x in {0/[, 1/1, 2/2, 3/-3, 4/0, 5/-4, 6/-1, 7/2, 8/]}
{
    \node[xshift=\s*25] (0,0) {$\x$};
}
}
\onslide<1-1>{
\node[xshift=0,yshift=-100] (0,0) {$[$};
\node[xshift=200,yshift=-100] (0,0) {$]$};
}
\onslide<2->{
\node[xshift=25,yshift=-100] (0,0) {$[$};
\node[xshift=175,yshift=-100] (0,0) {$]$};
}
\foreach \ss/\v in {2/-4,3/-2,4/-1,5/-1,6/6}
{
    \onslide<1->{ \node[xshift=\ss*25,yshift=-100] (0,0) {$\v$}; }
}
\end{tikzpicture}
\end{center}
\end{frame}


% \begin{frame}
%     \frametitle{Example of 1D Convolution: Valid padding}
%     \begin{equation*} \begin{split}
%     x &= \begin{bmatrix} 1 & 2 & -3 & 0 & 4 & -1 & 2 \end{bmatrix} \\
%     k &= \begin{bmatrix} -1 & 0 & 1 \end{bmatrix} \\
%     &\\
%     \onslide<1>{x * k &= \begin{bmatrix} ? & -4 & -2 & 7 & -1 & -2 & ? \end{bmatrix} \\}
%     \onslide<2->{x * k &= \begin{bmatrix} \quad & -4 & -2 & 7 & -1 & -2 & \quad \end{bmatrix} \\}
%     \end{split} \end{equation*}
% \end{frame}


\begin{frame}
    \frametitle{Example with Zero Padding}
    \begin{center}
\begin{tikzpicture}
\onslide<1-1>{
    \foreach \s/\x in {0/[, 1/1, 2/2, 3/-3, 4/0, 5/-4, 6/-1, 7/2, 8/]}
    {
        \node[xshift=\s*25] (0,0) {$\x$};
    }
    \node[xshift=0,yshift=-100] (0,0) {$[$};
    \node[xshift=200,yshift=-100] (0,0) {$]$};
}
\onslide<2->{
    \foreach \s/\x in {-1/[, 1/1, 2/2, 3/-3, 4/0, 5/-4, 6/-1, 7/2, 9/]}
    {
        \node[xshift=\s*25] (0,0) {$\x$};
    }
    \node[xshift=0,red] (0,0) {$0$};
    \node[xshift=200,red] (0,0) {$0$};
    \node[xshift=0,yshift=-100] (0,0) {$[$};
    \node[xshift=200,yshift=-100] (0,0) {$]$};
}
\foreach \ss/\v in {2/-4,3/-2,4/-1,5/-1,6/6}
{
    \onslide<1->{ \node[xshift=\ss*25,yshift=-100] (0,0) {$\v$}; }
}
\onslide<2->{
    \foreach \ss/\v in {1/2,7/1}
    {
            \node[xshift=(\ss-1)*25,yshift=-25] (0,0) {$-1$};
            \node[xshift=\ss*25,yshift=-25] (0,0) {$0$};
            \node[xshift=(\ss+1)*25,yshift=-25] (0,0) {$1$};
            \draw[|-|,below,xshift=(\ss-1)*25,yshift=-40] (0,0) -- (2,0) ;
            \draw[->,xshift=(\ss-1)*25,yshift=-40] (1,0) -- (1,-1) ;
            \node[xshift=\ss*25,yshift=-100] (0,0) {$\v$};
    }
}
\end{tikzpicture}
\end{center}
\end{frame}

%
% \begin{frame}
%     \frametitle{Example of 1D Convolution: Same padding}
%     \begin{equation*} \begin{split}
%     x &= \begin{bmatrix} 1 & 2 & -3 & 0 & 4 & -1 & 2 \end{bmatrix} \\
%     k &= \begin{bmatrix} -1 & 0 & 1 \end{bmatrix} \\
%     &\\
%     \onslide<1>{x * k &= \begin{bmatrix} ? & -4 & -2 & 7 & -1 & -2 & ? \end{bmatrix} \\}
%     \onslide<2->{x * k &= \begin{bmatrix} 2 & -4 & -2 & 7 & -1 & -2 & 1 \quad \end{bmatrix} \\}
%     \end{split} \end{equation*}
% \end{frame}


\begin{frame}
    \frametitle{Example with Periodic Padding}
    \begin{center}
\begin{tikzpicture}
\onslide<1-1>{
    \foreach \s/\x in {0/[, 1/1, 2/2, 3/-3, 4/0, 5/-4, 6/-1, 7/2, 8/]}
    {
        \node[xshift=\s*25] (0,0) {$\x$};
    }
    \node[xshift=0,yshift=-100] (0,0) {$[$};
    \node[xshift=200,yshift=-100] (0,0) {$]$};
}
\onslide<2->{
    \foreach \s/\x in {-1/[, 1/1, 2/2, 3/-3, 4/0, 5/-4, 6/-1, 7/2, 9/]}
    {
        \node[xshift=\s*25] (0,0) {$\x$};
    }
    \node[xshift=0,red] (0,0) {$2$};
    \node[xshift=200,red] (0,0) {$1$};
    \node[xshift=0,yshift=-100] (0,0) {$[$};
    \node[xshift=200,yshift=-100] (0,0) {$]$};
}
\foreach \ss/\v in {2/-4,3/-2,4/-1,5/-1,6/6}
{
    \onslide<1->{ \node[xshift=\ss*25,yshift=-100] (0,0) {$\v$}; }
}
\onslide<2->{
    \foreach \ss/\v in {1/0,7/2}
    {
            \node[xshift=(\ss-1)*25,yshift=-25] (0,0) {$-1$};
            \node[xshift=\ss*25,yshift=-25] (0,0) {$0$};
            \node[xshift=(\ss+1)*25,yshift=-25] (0,0) {$1$};
            \draw[|-|,below,xshift=(\ss-1)*25,yshift=-40] (0,0) -- (2,0) ;
            \draw[->,xshift=(\ss-1)*25,yshift=-40] (1,0) -- (1,-1) ;
            \node[xshift=\ss*25,yshift=-100] (0,0) {$\v$};
    }
}
\end{tikzpicture}
\end{center}
\end{frame}


%
% \begin{frame}
%     \frametitle{Example of 1D Convolution: Periodic padding}
%     \begin{equation*} \begin{split}
%     x &= \begin{bmatrix} 1 & 2 & -3 & 0 & 4 & -1 & 2 \end{bmatrix} \\
%     k &= \begin{bmatrix} -1 & 0 & 1 \end{bmatrix} \\
%     &\\
%     \onslide<1>{x * k &= \begin{bmatrix} ? & -4 & -2 & 7 & -1 & -2 & ? \end{bmatrix} \\}
%     \onslide<2->{x * k &= \begin{bmatrix} 0 & -4 & -2 & 7 & -1 & -2 & 2 \end{bmatrix} \\}
%     \end{split} \end{equation*}
% \end{frame}


\begin{frame}
    \frametitle{Convolutions in Neural Networks}
\begin{center}
    Learn small kernel instead of big weight matrix
\end{center}

\begin{figure}
\begin{tikzpicture}
\foreach \y in {0,1,2,3,4}
    \draw (0,\y) circle (0.2cm);
\foreach \y in {1,2,3}
    \draw (3,\y) circle (0.2cm);

\onslide<1-1>{
    \draw (0,1+1) --  node {-1} (3,1);
    \draw (0,1+0) --  node { 0} (3,1);
    \draw (0,1-1) --  node { 1}(3,1);
        }
\onslide<2->{
    \draw (0,1+1) --  (3,1);
    \draw (0,1+0) --  (3,1);
    \draw (0,1-1) --  (3,1);
        }
\onslide<2-2>{
    \draw (0,2+1) --  node {-1} (3,2);
    \draw (0,2+0) --  node { 0} (3,2);
    \draw (0,2-1) --  node { 1} (3,2);
        }
\onslide<3->{
    \draw (0,2+1) --  (3,2);
    \draw (0,2+0) --  (3,2);
    \draw (0,2-1) --  (3,2);
        }
\onslide<3->{
    \draw (0,3+1) --  node {-1} (3,3);
    \draw (0,3+0) --  node { 0} (3,3);
    \draw (0,3-1) --  node { 1} (3,3);
        }
\end{tikzpicture}
\end{figure}

\end{frame}



\begin{frame}
    \frametitle{Why CNNs help}
\begin{itemize}
    \item Position matters $\rightarrow$ model complexity
    \item Only consider local information $\rightarrow$ model complexity
    \item Only learn a small kernel, not big weight matrix $\rightarrow$ model complexity, size
\end{itemize}
\end{frame}


\begin{frame}
\frametitle{Problems where CNNs are useful}
\begin{itemize}
    \item image/video tasks
    \begin{itemize}
        \item classification
        \item segmentaion
        \item \dots
    \end{itemize}
    \item sequence analysis
    \item graphical models
\end{itemize}
\end{frame}


\begin{frame}
    \frametitle{Images: 2D convolution}
\begin{columns}
    \begin{column}{0.5\textwidth}
        \begin{center}
\begin{tikzpicture}
    \draw[fill=black!50] (0,4) -- (0,2.5) -- (2,2.5) -- (2,4) -- (0,4) ;
    \draw[fill=black!50] (0,2.5) -- (1,2.5) -- (1,0.5) -- (0,0.5) -- (0,2.5);
\foreach \x in {0,1,2,3,4,5,6}
    \draw (\x/2,0) -- (\x/2,4);
\foreach \y in {0,1,2,3,4,5,6,7,8}
    \draw (0,\y/2) -- (3,\y/2);
\end{tikzpicture}
\end{center}
\end{column}
\begin{column}{0.5\textwidth}
$\begin{bmatrix}
1 & 1 & 1 & 1 & 0 & 0 \\
1 & 1 & 1 & 1 & 0 & 0 \\
1 & 1 & 1 & 1 & 0 & 0 \\
1 & 1 & 0 & 0 & 0 & 0 \\
1 & 1 & 0 & 0 & 0 & 0 \\
1 & 1 & 0 & 0 & 0 & 0 \\
1 & 1 & 0 & 0 & 0 & 0 \\
0 & 0 & 0 & 0 & 0 & 0 \\
\end{bmatrix}$
\end{column}
\end{columns}
\end{frame}


\begin{frame}
    \frametitle{Images: 2D convolution}
\begin{columns}
    \begin{column}{0.25\textwidth}
        \begin{center}
\begin{tikzpicture}
    \draw[fill=black!50] (0,4) -- (0,2.5) -- (2,2.5) -- (2,4) -- (0,4) ;
    \draw[fill=black!50] (0,2.5) -- (1,2.5) -- (1,0.5) -- (0,0.5) -- (0,2.5);
\foreach \x in {0,1,2,3,4,5,6}
    \draw (\x/2,0) -- (\x/2,4);
\foreach \y in {0,1,2,3,4,5,6,7,8}
    \draw (0,\y/2) -- (3,\y/2);


\onslide<1-1>{
\draw[ultra thick] (0,0) -- (1.5,0) -- (1.5,1.5) -- (0,1.5) --(0,0) ;
\draw[ultra thick] (0.5,0) -- (0.5,1.5);
\draw[ultra thick] (0,0.5) -- (1.5,0.5);
\draw[ultra thick] (1,0) -- (1,1.5);
\draw[ultra thick] (0,1) -- (1.5,1);
}
\onslide<2-2>{
\draw[ultra thick] (0.5,0) -- (2,0) -- (2,1.5) -- (0.5,1.5) --(0.5,0) ;
\draw[ultra thick] (1,0) -- (1,1.5);
\draw[ultra thick] (0.5,0.5) -- (2,0.5);
\draw[ultra thick] (1.5,0) -- (1.5,1.5);
\draw[ultra thick] (0.5,1) -- (2,1);
}
\onslide<3-3>{
\draw[ultra thick] (1,0) -- (2.5,0) -- (2.5,1.5) -- (1,1.5) --(1,0) ;
\draw[ultra thick] (1.5,0) -- (1.5,1.5);
\draw[ultra thick] (1,0.5) -- (2.5,0.5);
\draw[ultra thick] (2,0) -- (2,1.5);
\draw[ultra thick] (1,1) -- (2.5,1);
}
\onslide<4-4>{
\draw[ultra thick] (1.5,0) -- (3,0) -- (3,1.5) -- (1.5,1.5) --(1.5,0) ;
\draw[ultra thick] (2,0) -- (2,1.5);
\draw[ultra thick] (1.5,0.5) -- (3,0.5);
\draw[ultra thick] (2.5,0) -- (2.5,1.5);
\draw[ultra thick] (1.5,1) -- (3,1);
}
\onslide<5-5>{
\draw[ultra thick] (0,0.5) -- (1.5,0.5) -- (1.5,2) -- (0,2) --(0,0.5) ;
\draw[ultra thick] (0.5,0.5) -- (0.5,2);
\draw[ultra thick] (0,1) -- (1.5,1);
\draw[ultra thick] (1,0.5) -- (1,2);
\draw[ultra thick] (0,1.5) -- (1.5,1.5);
}
\onslide<6-6>{
\draw[ultra thick] (0.5,0.5) -- (2,0.5) -- (2,2) -- (0.5,2) --(0.5,0.5) ;
\draw[ultra thick] (1,0.5) -- (1,2);
\draw[ultra thick] (0.5,1) -- (2,1);
\draw[ultra thick] (1.5,0.5) -- (1.5,2);
\draw[ultra thick] (0.5,1.5) -- (2,1.5);
}
\onslide<7-7>{
\draw[ultra thick] (1,0.5) -- (2.5,0.5) -- (2.5,2) -- (1,2) --(1,0.5) ;
\draw[ultra thick] (1.5,0.5) -- (1.5,2);
\draw[ultra thick] (1,1) -- (2.5,1);
\draw[ultra thick] (2,0.5) -- (2,2);
\draw[ultra thick] (1,1.5) -- (2.5,1.5);
}
\onslide<8-8>{
\draw[ultra thick] (1.5,0.5) -- (3,0.5) -- (3,2) -- (1.5,2) --(1.5,0.5) ;
\draw[ultra thick] (2,0.5) -- (2,2);
\draw[ultra thick] (1.5,1) -- (3,1);
\draw[ultra thick] (2.5,0.5) -- (2.5,2);
\draw[ultra thick] (1.5,1.5) -- (3,1.5);
}
\end{tikzpicture}
\end{center}
\end{column}
\begin{column}{0.1\textwidth}
    \begin{tikzpicture}
        \node (0,0) {{\footnotesize $* \begin{bmatrix} -1 & 0 & 1 \\ -2 & 0 & 2 \\ -1 & 0 & 1 \end{bmatrix} = $}};
\end{tikzpicture}
\end{column}
\begin{column}{0.5\textwidth}
    \begin{center}
    \begin{tikzpicture}
\onslide<1-1>{
    \node (0,0) {
    $\begin{bmatrix}
     &  &  &  &  &  \\
     &  &  &  &  &  \\
     &  &  &  & &  \\
     &  &  &  &  &  \\
     &  &  &  &  &  \\
     &  &  &  &  &  \\
     & -3 &  &  &  &  \\
     {\color{white}0}&{\color{white}0} &{\color{white}0}  &{\color{white}0}  &{\color{white}0}  &{\color{white}0}  \\
    \end{bmatrix}$
    };
}
\onslide<2-2>{
    \node (0,0) {
    $\begin{bmatrix}
     &  &  &  &  &  \\
     &  &  &  &  &  \\
     &  &  &  & &  \\
     &  &  &  &  &  \\
     &  &  &  &  &  \\
     &  &  &  &  &  \\
     & -3 & -3 &  &  &  \\
     {\color{white}0}&{\color{white}0} &{\color{white}0}  &{\color{white}0}  &{\color{white}0}  &{\color{white}0}  \\
    \end{bmatrix}$
    };
}
\onslide<3-3>{
    \node (0,0) {
    $\begin{bmatrix}
     &  &  &  &  &  \\
     &  &  &  &  &  \\
     &  &  &  & &  \\
     &  &  &  &  &  \\
     &  &  &  &  &  \\
     &  &  &  &  &  \\
     & -3 & -3 & 0 &  &  \\
     {\color{white}0}&{\color{white}0} &{\color{white}0}  &{\color{white}0}  &{\color{white}0}  &{\color{white}0}  \\
    \end{bmatrix}$
    };
}
\onslide<4-4>{
    \node (0,0) {
    $\begin{bmatrix}
     &  &  &  &  &  \\
     &  &  &  &  &  \\
     &  &  &  & &  \\
     &  &  &  &  &  \\
     &  &  &  &  &  \\
     &  &  &  &  &  \\
     & -3 & -3 & 0 & 0 &  \\
     {\color{white}0}&{\color{white}0} &{\color{white}0}  &{\color{white}0}  &{\color{white}0}  &{\color{white}0}  \\
    \end{bmatrix}$
    };
}
\onslide<5-5>{
    \node (0,0) {
    $\begin{bmatrix}
     &  &  &  &  &  \\
     &  &  &  &  &  \\
     &  &  &  & &  \\
     &  &  &  &  &  \\
     &  &  &  &  &  \\
     & -4 &  &  &  &  \\
     & -3 & -3 & 0 & 0 &  \\
     {\color{white}0}&{\color{white}0} &{\color{white}0}  &{\color{white}0}  &{\color{white}0}  &{\color{white}0}  \\
    \end{bmatrix}$
    };
}
\onslide<6-6>{
    \node (0,0) {
    $\begin{bmatrix}
     &  &  &  &  &  \\
     &  &  &  &  &  \\
     &  &  &  & &  \\
     &  &  &  &  &  \\
     &  &  &  &  &  \\
     & -4 & -4 &  &  &  \\
     & -3 & -3 & 0 & 0 &  \\
     {\color{white}0}&{\color{white}0} &{\color{white}0}  &{\color{white}0}  &{\color{white}0}  &{\color{white}0}  \\
    \end{bmatrix}$
    };
}
\onslide<7-7>{
    \node (0,0) {
    $\begin{bmatrix}
     &  &  &  &  &  \\
     &  &  &  &  &  \\
     &  &  &  & &  \\
     &  &  &  &  &  \\
     &  &  &  &  &  \\
     & -4 & -4 & 0 &  &  \\
     & -3 & -3 & 0 & 0 &  \\
     {\color{white}0}&{\color{white}0} &{\color{white}0}  &{\color{white}0}  &{\color{white}0}  &{\color{white}0}  \\
    \end{bmatrix}$
    };
}
\onslide<8-8>{
    \node (0,0) {
    $\begin{bmatrix}
     &  &  &  &  &  \\
     &  &  &  &  &  \\
     &  &  &  & &  \\
     &  &  &  &  &  \\
     &  &  &  &  &  \\
     & -4 & -4 & 0 & 0 &  \\
     & -3 & -3 & 0 & 0 &  \\
     {\color{white}0}&{\color{white}0} &{\color{white}0}  &{\color{white}0}  &{\color{white}0}  &{\color{white}0}  \\
    \end{bmatrix}$
    };
}
\onslide<9-9>{
    \node (0,0) {
    $\begin{bmatrix}
     &  &  &  &  &  \\
     & 0 & 0 & -4 & -4 &  \\
     & -1 & -1 & -3 & -3 &  \\
     & -3 & -3 & -1 & -1 &  \\
     & -4 & -4 & 0 & 0 &  \\
     & -4 & -4 & 0 & 0 &  \\
     & -3 & -3 & 0 & 0 &  \\
     {\color{white}0}&{\color{white}0} &{\color{white}0}  &{\color{white}0}  &{\color{white}0}  &{\color{white}0}  \\
    \end{bmatrix}$
    };
}
\end{tikzpicture}
\end{center}
\end{column}
\end{columns}
\end{frame}


\begin{frame}
\frametitle{Channels in, channels out}
\begin{itemize}
    % \item instead of keeping track of each pixel, treat each learned image feature
    \item keep track of \emph{image feature} instead of individual pixels
    \item node in NN is a channel ( = feature )
    \item add bias to every channel \\\qquad\qquad$\rightarrow$ add constant to each image feature
    \item apply $\sigma$ to each pixel
    \item channels in $\rightarrow$ \# convolutions summed together in output
    \item channels out $\rightarrow$ \# distinct convolution kernels learned
\end{itemize}
% \begin{center}
% node = pixel
% ~\\~\\
% node = image
% \end{center}
\end{frame}

%
% \begin{frame}
% \frametitle{Multiple channels in}
% \begin{columns}
%     \begin{column}{0.5\textwidth}
%         \begin{center}
% \begin{tikzpicture}
%
% \foreach \color/\shift in {blue/0.8, green/0.4, red/0}
% {
%     \draw[fill=white] (0-\shift,0+\shift) -- (3-\shift,0+\shift) -- (3-\shift,4+\shift) -- (0-\shift,4+\shift) -- (0-\shift,0+\shift);
%     \draw[fill=\color!50] (0-\shift,4+\shift) -- (0-\shift,2.5+\shift) -- (2-\shift,2.5+\shift) -- (2-\shift,4+\shift) -- (0-\shift,4+\shift) ;
%     \draw[fill=\color!50] (0-\shift,2.5+\shift) -- (1-\shift,2.5+\shift) -- (1-\shift,0.5+\shift) -- (0-\shift,0.5+\shift) -- (0-\shift,2.5+\shift);
%     \foreach \x in {0,1,2,3,4,5,6}
%         \draw (\x/2 -\shift,0+\shift) -- (\x/2 -\shift,4+\shift);
%     \foreach \y in {0,1,2,3,4,5,6,7,8}
%         \draw (0-\shift,\y/2 +\shift) -- (3-\shift,\y/2 +\shift);
% }
%
%     \end{tikzpicture}
%     \end{center}
%     \end{column}
%     \begin{column}{0.5\textwidth}
% \begin{tikzpicture}
% \foreach \c/\shift/\cc in {blue/20/white, green/10/white, red/0/red}
% {
%     \node[xshift=-\shift,yshift=\shift] (0,0) {\color{\c}
%     $\begin{bmatrix}
%     {\color{\cc} 1} & {\color{\cc} 1} & {\color{\cc} 1} & {\color{\cc} 1} &{\color{\cc} 0}& {\color{\cc} 0} \\
%     {\color{\cc} 1} & {\color{\cc} 1} & {\color{\cc} 1} & {\color{\cc} 1} &{\color{\cc} 0}& {\color{\cc} 0} \\
%     {\color{\cc} 1} & {\color{\cc} 1} & {\color{\cc} 1} & {\color{\cc} 1} &{\color{\cc} 0}& {\color{\cc} 0} \\
%     {\color{\cc} 1} & {\color{\cc} 1} & {\color{\cc} 0} & {\color{\cc} 0} &{\color{\cc} 0}& {\color{\cc} 0} \\
%     {\color{\cc} 1} & {\color{\cc} 1} & {\color{\cc} 0} & {\color{\cc} 0} &{\color{\cc} 0}& {\color{\cc} 0} \\
%     {\color{\cc} 1} & {\color{\cc} 1} & {\color{\cc} 0} & {\color{\cc} 0} &{\color{\cc} 0}& {\color{\cc} 0} \\
%     {\color{\cc} 1} & {\color{\cc} 1} & {\color{\cc} 0} & {\color{\cc} 0} &{\color{\cc} 0}& {\color{\cc} 0} \\
%     {\color{\cc} 0} & {\color{\cc} 0} & {\color{\cc} 0} & {\color{\cc} 0} &{\color{\cc} 0}& {\color{\cc} 0} \\
%     \end{bmatrix}$
%     };
%     }
% \end{tikzpicture}
% \end{column}
% \end{columns}
% \end{frame}


\begin{frame}
\frametitle{Convolution with multiple channels in}
\begin{columns}
    \begin{column}{0.13\textwidth}
        \begin{center}
\begin{tikzpicture}

\foreach \color/\shift in {blue/0.8, green/0.4, red/0}
{
    \draw[fill=white] (0-\shift,0+\shift) -- (3-\shift,0+\shift) -- (3-\shift,4+\shift) -- (0-\shift,4+\shift) -- (0-\shift,0+\shift);
    \draw[fill=\color!50] (0-\shift,4+\shift) -- (0-\shift,2.5+\shift) -- (2-\shift,2.5+\shift) -- (2-\shift,4+\shift) -- (0-\shift,4+\shift) ;
    \draw[fill=\color!50] (0-\shift,2.5+\shift) -- (1-\shift,2.5+\shift) -- (1-\shift,0.5+\shift) -- (0-\shift,0.5+\shift) -- (0-\shift,2.5+\shift);
    \foreach \x in {0,1,2,3,4,5,6}
        \draw (\x/2 -\shift,0+\shift) -- (\x/2 -\shift,4+\shift);
    \foreach \y in {0,1,2,3,4,5,6,7,8}
        \draw (0-\shift,\y/2 +\shift) -- (3-\shift,\y/2 +\shift);
}
\end{tikzpicture}
\end{center}
\end{column}
\begin{column}{0.5\textwidth}
\begin{tikzpicture}
\foreach \c/\shift in {blue/20, green/10, red/0}
    {
        \draw[xshift=-\shift,yshift=\shift,ultra thick,fill=\c!50] (0,0) -- (0,1) -- (1,1) -- (1,0) -- (0,0);
        \draw[xshift=-\shift,yshift=\shift,ultra thick] (1/3,0) -- (1/3,1);
        \draw[xshift=-\shift,yshift=\shift,ultra thick] (2/3,0) -- (2/3,1);
        \draw[xshift=-\shift,yshift=\shift,ultra thick] (0,1/3) -- (1,1/3);
        \draw[xshift=-\shift,yshift=\shift,ultra thick] (0,2/3) -- (1,2/3);
    }
\node[xshift=-50,yshift=20] (0,0) {{\huge $*$}};
\draw[xshift=50,yshift=20,->,ultra thick] (0,0) -- (1.3,0) ;
\node[xshift=110,yshift=20] (0,0) {{\large sum}};
% \foreach \color/\shift/\yshft in {black/20/0}
% {
%     \draw[fill=white] (0-\shift,0+\yshft) -- (3-\shift,0+\yshft) -- (3-\shift,4+\yshft) -- (0-\shift,4+\yshft) -- (0-\shift,0+\yshft);
%     \draw[fill=\color!50] (0-\shift,4+\yshft) -- (0-\shift,2.5+\yshft) -- (2-\shift,2.5+\yshft) -- (2-\shift,4+\yshft) -- (0-\shift,4+\yshft) ;
%     \draw[fill=\color!50] (0-\shift,2.5+\yshft) -- (1-\shift,2.5+\yshft) -- (1-\shift,0.5+\yshft) -- (0-\shift,0.5+\yshft) -- (0-\shift,2.5+\yshft);
%     \foreach \x in {0,1,2,3,4,5,6}
%         \draw (\x/2 -\shift,0+\yshft) -- (\x/2 -\shift,4+\yshft);
%     \foreach \y in {0,1,2,3,4,5,6,7,8}
%         \draw (0-\shift,\y/2 +\yshft) -- (3-\shift,\y/2 +\yshft);
% }
\end{tikzpicture}
\end{column}
\end{columns}
\end{frame}


\begin{frame}
    \frametitle{Convolution with multiple channels in}
{\normalsize $$ (320 \times 240 \times 3) \,\, * \,\,  3 \times 3 \times 3\text{ kernel} \,\, = \,\, (320 \times 240 \times 1) $$ }
\end{frame}

\begin{frame}
    \frametitle{Convolution with multiple channels in}
{\normalsize $$ (n_x \times n_y \times n_{in}) \,\, * \,\,  n_k \times n_k \times n_{in} \text{ kernel} \,\, = \,\, (n_x \times n_y \times 1) $$ }
~\\
\begin{columns}[t]
    \begin{column}{0.5\textwidth}
        \begin{itemize}
            \item $n_x$ : pixels in $x$ direction
            \item $n_y$ : pixels in $y$ direction
            \item $n_k$ : window size of convolution kernel
        \end{itemize}
    \end{column}
    \begin{column}{0.5\textwidth}
        \begin{itemize}
            \item $n_{in}$ \,: channels in
            \item $n_{out}$ : channels out
        \end{itemize}
    \end{column}
\end{columns}

\end{frame}


\begin{frame}
    \frametitle{Convolution with multiple channels out}
    {\normalsize $$ (n_x \times n_y \times n_{in}) \,\, * \,\,  [\, n_k \times n_k \times n_{in} \, ]\times n_{out} \text{ kernel} \,\, = \,\, (n_x \times n_y \times n_{out}) $$ }
    ~\\
    \begin{columns}[t]
        \begin{column}{0.5\textwidth}
            \begin{itemize}
                \item $n_x$ : pixels in $x$ direction
                \item $n_y$ : pixels in $y$ direction
                \item $n_k$ : window size of convolution kernel
            \end{itemize}
        \end{column}
        \begin{column}{0.5\textwidth}
            \begin{itemize}
                \item $n_{in}$ \,: channels in
                \item $n_{out}$ : channels out
            \end{itemize}
        \end{column}
    \end{columns}

\end{frame}


\begin{frame}
    \frametitle{A complete convolution layer}
$$ \text{layer}_\ell \longmapsto \vec{\sigma} \left(\,\, \text{layer}_{\ell} \,\,*\,\, \text{kernel}_{\ell} \,\,+ \text{bias}_\ell \,\,\right) = \text{layer}_{\ell+1}$$
\end{frame}


\begin{frame}
    \frametitle{Downsides}
\begin{itemize}
    \item Longer to train \\\qquad $\rightarrow$ backpropagation needs global data \\\qquad {\color{white}$\rightarrow$} to update kernel weights \\ \qquad $\rightarrow$ generalization: locally connected layers ~\\~\\
    \item Requires a grid \\\qquad $\rightarrow$ generalization: graph convolutions
\end{itemize}
\end{frame}

\section{Multi-resolution Networks}
\begin{frame}
    \frametitle{Data at multiple scales}
\begin{itemize}
    \item images
    \item time series data
\end{itemize}
\end{frame}


\begin{frame}
    \frametitle{Downsampling and Upsampling}
\begin{itemize}
    \item Reduces/increases number of pixels at each node
    \item Captures data at different resolutions
    \item Removes noise
\end{itemize}
\end{frame}


\begin{frame}
    \frametitle{Example: Downsampling via Max Pooling}
    \begin{columns}
        \begin{column}{0.5\textwidth}
            \begin{center}
    \begin{tikzpicture}
        \draw[fill=black!20] (0,4) -- (0,2.5) -- (2,2.5) -- (2,4) -- (0,4) ;
        \draw[fill=black!50] (0,2.5) -- (1,2.5) -- (1,0.5) -- (0,0.5) -- (0,2.5);
        \draw[ultra thick] (0,3) -- (0,2) -- (1,2) -- (1,3) -- (0,3);
        \draw[ultra thick] (0,2.5) -- (1,2.5);
        \draw[ultra thick] (0.5,2) -- (0.5,3);
    \foreach \x in {0,1,2,3,4,5,6}
        \draw (\x/2,0) -- (\x/2,4);
    \foreach \y in {0,1,2,3,4,5,6,7,8}
        \draw (0,\y/2) -- (3,\y/2);
    \end{tikzpicture}
    \end{center}
    \end{column}
        \begin{column}{0.5\textwidth}
            \begin{center}
    \begin{tikzpicture}
        \draw[fill=black!20] (0,4) -- (2,4) -- (2,2) -- (1,2) --(1,3) --(0,3) -- (0,4) ;
        \draw[fill=black!50] (0,3) -- (1,3) -- (1,0) -- (0,0) -- (0,3);
        \draw[ultra thick] (0,3) -- (0,2) -- (1,2) -- (1,3) -- (0,3);
    \foreach \x in {0,1,2,3}
        \draw (\x,0) -- (\x,4);
    \foreach \y in {0,1,2,3,4}
        \draw (0,\y) -- (3,\y);
    \end{tikzpicture}
    \end{center}
    \end{column}
    \end{columns}

\end{frame}


\begin{frame}
    \frametitle{Example: Upsampling}
    \begin{columns}
    \begin{column}{0.5\textwidth}
        \begin{center}
\begin{tikzpicture}
    \draw[fill=black!20] (0,4) -- (2,4) -- (2,2) -- (1,2) --(1,3) --(0,3) -- (0,4) ;
    \draw[fill=black!50] (0,3) -- (1,3) -- (1,0) -- (0,0) -- (0,3);
    \draw[ultra thick] (0,3) -- (0,2) -- (1,2) -- (1,3) -- (0,3);
\foreach \x in {0,1,2,3}
    \draw (\x,0) -- (\x,4);
\foreach \y in {0,1,2,3,4}
    \draw (0,\y) -- (3,\y);
\end{tikzpicture}
\end{center}
\end{column}
\begin{column}{0.5\textwidth}
     \begin{center}
\begin{tikzpicture}
    \draw[fill=black!20] (0,4) -- (2,4) -- (2,2) -- (1,2) --(1,3) --(0,3) -- (0,4) ;
    \draw[fill=black!50] (0,3) -- (1,3) -- (1,0) -- (0,0) -- (0,3);
    \draw[ultra thick] (0,3) -- (0,2) -- (1,2) -- (1,3) -- (0,3);
    \draw[ultra thick] (0,2.5) -- (1,2.5);
    \draw[ultra thick] (0.5,2) -- (0.5,3);
\foreach \x in {0,1,2,3,4,5,6}
 \draw (\x/2,0) -- (\x/2,4);
\foreach \y in {0,1,2,3,4,5,6,7,8}
 \draw (0,\y/2) -- (3,\y/2);
\end{tikzpicture}
\end{center}
\end{column}
\end{columns}

\end{frame}

\begin{frame}
    \frametitle{Pooling causes loss of information}
    \begin{columns}
    \begin{column}{0.5\textwidth}
        \begin{center}
            \begin{tikzpicture}
                \draw[fill=black!20] (0,4) -- (0,2.5) -- (2,2.5) -- (2,4) -- (0,4) ;
                \draw[fill=black!50] (0,2.5) -- (1,2.5) -- (1,0.5) -- (0,0.5) -- (0,2.5);
                % \draw[ultra thick] (0,3) -- (0,2) -- (1,2) -- (1,3) -- (0,3);
                % \draw[ultra thick] (0,2.5) -- (1,2.5);
                % \draw[ultra thick] (0.5,2) -- (0.5,3);
            \foreach \x in {0,1,2,3,4,5,6}
                \draw (\x/2,0) -- (\x/2,4);
            \foreach \y in {0,1,2,3,4,5,6,7,8}
                \draw (0,\y/2) -- (3,\y/2);
            \end{tikzpicture} ~\\
            Original
\end{center}
\end{column}
\begin{column}{0.5\textwidth}
     \begin{center}
\begin{tikzpicture}
    \draw[fill=black!20] (0,4) -- (2,4) -- (2,2) -- (1,2) --(1,3) --(0,3) -- (0,4) ;
    \draw[fill=black!50] (0,3) -- (1,3) -- (1,0) -- (0,0) -- (0,3);
    % \draw[ultra thick] (0,3) -- (0,2) -- (1,2) -- (1,3) -- (0,3);
    % \draw[ultra thick] (0,2.5) -- (1,2.5);
    % \draw[ultra thick] (0.5,2) -- (0.5,3);
\foreach \x in {0,1,2,3,4,5,6}
 \draw (\x/2,0) -- (\x/2,4);
\foreach \y in {0,1,2,3,4,5,6,7,8}
 \draw (0,\y/2) -- (3,\y/2);
\end{tikzpicture} ~\\
Downsampled and upsampled
\end{center}
\end{column}
\end{columns}

\end{frame}

\begin{frame}
    \frametitle{Usage}
\begin{tikzpicture}
    \foreach \s in {0,5,10}
    {
    \draw[fill=white,xshift=\s,yshift=-\s] (0,1) -- (1,1) -- (1,2) -- (0,2) -- (0,1);
    \draw[xshift=\s,yshift=-\s] (0.5,1) -- (0.5,2);
    \draw[xshift=\s,yshift=-\s] (0.25,1) -- (0.25,2);
    \draw[xshift=\s,yshift=-\s] (0.75,1) -- (0.75,2);
    \draw[xshift=\s, yshift=-\s] (1,1.5) -- (0,1.5);
    \draw[xshift=\s, yshift=-\s] (1,1.25) -- (0,1.25);
    \draw[xshift=\s, yshift=-\s] (1,1.75) -- (0,1.75);
    \draw[fill=white,xshift=\s,yshift=-\s] (3,1) -- (4,1) -- (4,2) -- (3,2) -- (3,1);
    \draw[xshift=\s,yshift=-\s] (3.5,1) -- (3.5,2);
    \draw[xshift=\s,yshift=-\s] (3.25,1) -- (3.25,2);
    \draw[xshift=\s,yshift=-\s] (3.75,1) -- (3.75,2);
    \draw[xshift=\s, yshift=-\s] (3,1.5) -- (4,1.5);
    \draw[xshift=\s, yshift=-\s] (3,1.25) -- (4,1.25);
    \draw[xshift=\s, yshift=-\s] (3,1.75) -- (4,1.75);
    \draw[fill=white,xshift=\s,yshift=-\s] (3,-2) -- (4,-2) -- (4,-1) -- (3,-1) -- (3,-2);
    \draw[xshift=\s,yshift=-\s] (3.5,-1) -- (3.5,-2);
    \draw[xshift=\s, yshift=-\s] (3,-1.5) -- (4,-1.5);
}
\draw[->] (1.5,1.5) -- (2.5,1.5);
\draw[->,above left] (2,1.5) -- (2,-1.5) node {{\small MaxPool}}-- (2.5,-1.5);
\draw[->,above] (4.5,1.5) -- node {{\small Conv2D}} (9.5,1.5) ;
\draw[->,above] (4.5,-1.5) -- node {{\small Conv2D}} (7.5,-1.5) ;
\foreach \s in {0,5,10,15}
    {
    \draw[fill=white,xshift=\s,yshift=-\s] (8,-2) -- (9,-2) -- (9,-1) -- (8,-1) -- (8,-2);
    \draw[xshift=\s,yshift=-\s] (8.5,-1) -- (8.5,-2);
    \draw[xshift=\s, yshift=-\s] (8,-1.5) -- (9,-1.5);
}
\draw[->,above right] (10,-1.5) -- (11.5,-1.5) node {{\small Upsample}}-- (11.5,-0.5);
\foreach \s in {0,5,10,15,20,25,30}
    {
\draw[fill=white,xshift=\s,yshift=-\s] (10,1) -- (11,1) -- (11,2) -- (10,2) -- (10,1);
\draw[xshift=\s,yshift=-\s] (10.5,1) -- (10.5,2);
\draw[xshift=\s,yshift=-\s] (10.25,1) -- (10.25,2);
\draw[xshift=\s,yshift=-\s] (10.75,1) -- (10.75,2);
\draw[xshift=\s,yshift=-\s] (11,1.5) -- (10,1.5) ;
\draw[xshift=\s,yshift=-\s] (11,1.25) -- (10,1.25) ;
\draw[xshift=\s,yshift=-\s] (11,1.75) -- (10,1.75) ;
}
\draw (9.75,2) -- (9.75,1) -- (10.2,0.55);
\draw (10.3,0.45) -- (11,-0.25) -- (12,-0.25);
\end{tikzpicture}

\end{frame}



\begin{frame}
    \frametitle{Concerns}
    \begin{itemize}
\item aliasing : lose high frequencies
\item loss of information : downsample + upsample $\ne$ original
\end{itemize}
\end{frame}


\section{Dropout}
\begin{frame}
    \frametitle{Why use Dropout}
\begin{itemize}
    \item Form of regularization \\ \qquad\qquad $\rightarrow$ force net to learn redundancies ~\\~\\
    \item Protects against overfitting \\\qquad\qquad $\rightarrow$ better generalization
\end{itemize}
\end{frame}


\begin{frame}
    \frametitle{What is Dropout?}
\begin{enumerate}
    \item Before training, fix probability $p$
    \item Each epoch, take $z_i \sim$ Ber$(p)$ for each output channel $i$
    \item If $z_i = 1$, update channel $i$ during this epoch
    \item Else, ignore channel $i$ during this epoch
\end{enumerate}
\end{frame}

\begin{frame}
    \frametitle{What is Dropout?}
\begin{equation*} \begin{split}
z_i &\sim \text{Ber}(p) \quad \text{for each channel }i \text{ in layer output} \\
Z &= \text{diag}(z_i) \\
&\\
x &\mapsto Z \vec{\sigma}(W x + b) \\
\end{split} \end{equation*}
\end{frame}

\begin{frame}
    \frametitle{Notes}
\begin{itemize}
    \item Each epoch, a random subset of channels is updated
    \item At evaluation, all channels are used
    \item Drop channels, not individual weights
    \begin{itemize}
        \item Feedforward NN : drop neurons
        \item Convolutional NN : drop channels
    \end{itemize}
\end{itemize}
\end{frame}


\begin{frame}
    \frametitle{Changes to training}
\begin{itemize}
    \item Requires more steps of gradient descent
    \item Fewer parameters to update at every step \\ \qquad\qquad $\rightarrow$ empirically trains faster ~\\~\\
    \item No extra parameters to learn
\end{itemize}
\end{frame}



\section{Batch Normalization}
\begin{frame}
    \frametitle{Batch Normalization}
    \begin{itemize}
        \item combat vanishing and exploding gradients
        \item assume each layer maintains same relative scale + distribution
        \item fight \emph{covariance shift}
    \end{itemize}
\end{frame}


\begin{frame}
    \frametitle{Details}
Normalize each channel individually, then rescale
\begin{enumerate}
    \item $\mu_B$ = batch mean
    \item $\sigma_B$ = batch variance
    \item Normalize each channel $x \longmapsto \frac{x - \mu_B}{\sigma_B} := \hat{x}$
    \item Affine transformation $\hat{x} \longmapsto \gamma \, \hat{x} + \beta$
\end{enumerate}
\end{frame}

\begin{frame}
    \frametitle{Changes to training}
\begin{itemize}
    \item Allows (empirically) use of bigger learning rates \\
    \qquad\qquad $\rightarrow$ easier to train ~\\~\\
    \item Backpropagation to update $\gamma,\,\beta$
\end{itemize}
\end{frame}
%
% %***********************************************************
%
\begin{frame}{Summary}
\begin{columns}[t]
    \begin{column}{0.5\textwidth}
\begin{itemize}
\item convolutions
\item upsampling and downsampling
\item dropout
\item batch normalization
\end{itemize}
\end{column}
\begin{column}{0.5\textwidth}
\begin{itemize}
    \item combat model complexity
    \item improve generalization
    \item overcome difficulties in training
\end{itemize}
\end{column}
\end{columns}
\end{frame}
% %***********************************************************
\end{document}
